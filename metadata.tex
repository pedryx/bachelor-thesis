

%%% Choose a language %%%

\newif\ifEN
% \ENtrue   % uncomment this for english
\ENfalse   % uncomment this for czech

%%% Configuration of the title page %%%

\def\ThesisTitleStyle{mff} % MFF style
%\def\ThesisTitleStyle{cuni} % uncomment for old-style with cuni.cz logo
%\def\ThesisTitleStyle{natur} % uncomment for nature faculty logo

\def\UKFaculty{Faculty of Mathematics and Physics}
%\def\UKFaculty{Faculty of Science}

\def\UKName{Charles University in Prague} % this is not used in the "mff" style

% Thesis type names, as used in several places in the title
\def\ThesisTypeTitle{\ifEN BACHELOR THESIS \else BAKALÁŘSKÁ PRÁCE \fi}
%\def\ThesisTypeTitle{\ifEN MASTER THESIS \else DIPLOMOVÁ PRÁCE \fi}
%\def\ThesisTypeTitle{\ifEN RIGOROUS THESIS \else RIGORÓZNÍ PRÁCE \fi}
%\def\ThesisTypeTitle{\ifEN DOCTORAL THESIS \else DISERTAČNÍ PRÁCE \fi}
\def\ThesisGenitive{\ifEN bachelor \else bakalářské \fi}
%\def\ThesisGenitive{\ifEN master \else diplomové \fi}
%\def\ThesisGenitive{\ifEN rigorous \else rigorózní \fi}
%\def\ThesisGenitive{\ifEN doctoral \else disertační \fi}
\def\ThesisAccusative{\ifEN bachelor \else bakalářskou \fi}
%\def\ThesisAccusative{\ifEN master \else diplomovou \fi}
%\def\ThesisAccusative{\ifEN rigorous \else rigorózní \fi}
%\def\ThesisAccusative{\ifEN doctoral \else disertační \fi}



%%% Fill in your details %%%

% (Note: \xxx is a "ToDo label" which makes the unfilled visible. Remove it.)
\def\ThesisTitle{Možnosti návrhového vzoru Entity-Component-System: případová studie}
\def\ThesisAuthor{Petr Kotáb}
\def\YearSubmitted{2024}

% department assigned to the thesis
\def\Department{Katedra softwaru a výuky informatiky}
% Is it a department (katedra), or an institute (ústav)?
\def\DeptType{Katedra}

\def\Supervisor{Mgr. Pavel Ježek, Ph.D.}
\def\SupervisorsDepartment{Katedra distribuovaných a spolehlivých systémů}

% Study programme and specialization
\def\StudyProgramme{Informatika }

\def\Dedication{%
Chtěl bych především poděkovat mému vedoucímu, Mgr.~Pavlu~Ježkovi,~Ph.D., za čas, připomínky a rady, které mi věnoval, bez nichž by tato práce nemohla vzniknout. Také děkuji svému původnímu vedoucímu, Mgr.~Jakubovi~Gemrotovi,~Ph.D., za čas a konzultace. Dále bych chtěl poděkovat svým přátelům a rodině za jejich velkou podporu.
}

\def\AbstractEN{
The thesis focuses on measuring the performance of ECS libraries for the C\# programming language. Unlike the often conducted simple tests, the goal of this thesis is to perform more complex measurements on a sample game. The result of this thesis is a sample non-interactive game simulating villagers harvesting resources in an open world. To enable the measurement of ECS libraries performance, an abstraction layer was prepared, allowing a single game implementation to launch with different ECS libraries. Before the actual measurement, we categorized the individual ECS libraries and formulated a hypothesis regarding the expected performance for these categories. In the end, we conducted a series of measurements that allowed us to confirm our hypothesis.
}

% This abstraction layer was prepared for a wide range of ECS libraries, so they can be included in our measurements.

\def\AbstractCS{
Práce se zabývá měřením výkonu ECS knihoven pro programovací jazyk C\#. Na rozdíl od často prováděných jednoduchých testů je cílem této práce provést komplexnější měření na ukázkové hře. Výsledkem práce je ukázková neinteraktivní hra simulující vesničany těžící suroviny v otevřeném světě. Pro umožnění měření výkonu ECS knihoven byla připravena abstrakční vrstva, která umožňuje jednu implementaci hry spouštět nad různými ECS knihovnami. Před samotným měřením jsme jednotlivé ECS knihovny rozdělili do kategorií a pro tyto kategorie stanovili hypotézu stran očekávané výkonnosti. Na závěr jsme provedli sadu měření, kterými se nám podařilo naši hypotézu potvrdit.
}

% Tuto abstrakční vrstvu jsme připravili pro širokou škálu ECS knihoven, které tím pádem mohou být součástí našeho měření.

% 3 to 5 keywords (recommended), each enclosed in curly braces.
% Keywords are useful for indexing and searching for the theses by topic.
\def\Keywords{
    {návrhový vzor}
    {Entity-Component-System}
    {počítačové hry}
    {simulace}
    {případová studie}
}

% If your abstracts are long and do not fit in the infopage, you can make the
% fonts a bit smaller by this setting. (Also, you should try to compress your abstract more.)
% Alternatively, consider increasing the size of the page by uncommenting the
% geometry modification in thesis.tex.
%\def\InfoPageFont{}
\def\InfoPageFont{\small}  %uncomment to decrease font size

\ifEN\relax\else
% If you are writing a czech thesis, you additionally need to fill in the
% english translation of the metadata here!
\def\ThesisTitleEN{Exploring Options of Entity-Component-System Design Pattern: A Case Study}
\def\DepartmentEN{Department of software and computer science education}
\def\DeptTypeEN{Department}
\def\SupervisorsDepartmentEN{Department of Distributed and Dependable Systems}
\def\StudyProgrammeEN{Computer Science}
\def\KeywordsEN{
    {design pattern}
    {Entity-Component-System}
    {computer games}
    {simulation}
    {case study}
}
\fi