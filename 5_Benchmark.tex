\chapter{Měření}
\label{chap:benchmark}
Náš projekt se skládá ze dvou částí a to hry a poté měření, které využívá vytvořenou hru pro měření výkonu jednotlivých ECS knihoven. Analýzu tohoto měření jsme provedli v sekci~\ref{benchmark-analysis} a jeho implementaci jsme popsali v sekci~\ref{benchmark-implementation}. V této kapitole si nejprve stanovíme hypotézu, poté si přiblížíme ECS knihovny, které budeme měřit a na závěr si prezentujeme výsledky tohoto měření.

\section{Hypotéza}
\label{chap:hypothesis}
V této sekci si stanovíme hypotézu našeho měření. V našem měření měříme jak dlouho trvá odsimulovat určitý počet iterací naší hry s jednotlivými ECS knihovnami. Ty se dají rozdělit na kategorie podle určitých vlastností. V této kapitole stanovíme jak si myslíme, že se jednotlivé kategorie umístí.

% Do teď jsme se věnovali samotné hře, kterou v následujících kapitolách budeme chtít použít pro porovnání výkonu ECS knihoven. Ale ještě předtím, konkrétně v této kapitole, si rozebereme předpokládané výsledky tohoto porovnání.

\subsection{Cache}
Využití cache je klíčový faktor pro výkon ECS knihovny, proto si nyní připomene co to vlastně je a jak funguje.

Program si svoje data uchovává v paměti RAM. Ovšem přístup do této paměti je z hlediska času drahý. Pro představu u moderních pamětí takovíto přístup může trvat vyšší desítky nanosekund. Z tohoto důvodu se využívají menší paměti, takzvané cache, které disponují mnohem vyšší rychlostí ale mnohem menší velikostí. Pro představu moderní procesory používají cache, u kterých přístup může trvat menší desetiny nanosekund.

Při přístupu k datům se procesor nejprve podívá zda nemá data již v této cache. Pokud ano, jedná se o takzvaný \textit{cache hit} a data si z ní načte. Pokud ne, jedná se o takzvaný \textit{cache miss} a je nutné data načíst z hlavní paměti. 

Již víme, že \textit{cache miss} jsou drahé, proto je potřeba pro vysoký výkon je minimalizovat. Ale jak přesně procesor rozhoduje o tom, jaká data budou v cache a jaká ne. Využívá se takzvané \textit{časové} a \textit{prostorové lokality}. \textit{Časová lokalita} spočívá v tom, že pokud jsme přistoupili k nějakým datům, je velká šance, že k nim brzo budeme chtít přistoupit znovu. Nadruhou stranu \textit{prostorová lokalita} spočívá v tom, že pokud jsme přistoupili k nějakým datům, je velká šance, že budeme chtít přistoupit také k datům, které se nacházejí blízko nich.

Jedna z možností, jak minimalizovat počet \textit{cache miss}, je využít sekvenčního přístupu. Pokud budeme přistupovat k datům, které jsou v paměti hned za sebou, tak díky \textit{prostorové lokalitě} bude počet \textit{cache miss} velmi malí.

Je nutné upozornit, že výše popsaný model cache je úmyslně zjednodušen. Velice detailní popis cache lze nalézt v \textit{Cache Memories}~\cite{10.1145/356887.356892} od \textit{Alana Jaye Smitha}. Stručnější popis a také informace o tom jak lze cache využít lze nalézt v již zmiňované knize \textit{Game Programming Patterns}~\cite{nystrom2014game}, konkrétně v kapitole \textit{Data Locality}. 

\subsection{Arch type}
Některé ECS knihovny pro lepší výkon používají \textit{arch type}. Jedná se o datovou strukturu pro ukládání komponent. Knihovny, které používají tuto datovou strukturu mají většinou velký výkon, proto před stanovením hypotézy si použití \textit{arch type} lehce přiblížíme.

Jak již bylo zmíněno, \textit{arch type} je datová struktura. Tato datová struktura představuje typ entity. Tento typ je jednoznačně určen typy všech komponent dané entity. Je možné si jej představit jako tabulku, kde jednotlivé sloupečky odpovídají komponentám a v každém řádku se nacházejí instance komponent pro danou entitu. Například můžeme mít \textit{arch type} vyobrazený v tabulce~\ref{tab:arch-type}. Tento \textit{arch type} je jednoznačně určen typy komponent \texttt{Position} (představující pozici entity), \texttt{Health} (představující počet životů entity) a \texttt{Damage} (představující poškození, které entita může udělit jiné entitě).

\begin{table}[!htb]
    \centering\footnotesize\sf
    \begin{tabular}{c c c c}
        \toprule
        entita & \texttt{Position} & \texttt{Health} & \texttt{Damage} \\
        \midrule
        hráč & $(0,0)$ & 10 & 3 \\
        nepřítel & (5,4) & 6 & 1 \\
        npc & (-4,8) & 16 & 9\\
        \bottomrule
    \end{tabular}
    \caption{Tabulka vyobrazující \textit{arch type}, který je jednoznačně určen komponentami \texttt{Position}, \texttt{Health} a \texttt{Damage}. Tento \textit{arch type} obsahuje tři entity, konkrétně hráče, nepřítele a npc.}
    \label{tab:arch-type}
\end{table}

Jednotlivé \textit{arch type} jsou uloženy ve \textit{world}. Každý \textit{arch type} má v sobě několik polí, konkrétně jedno pole pro každý typ komponenty. V každém poli jsou poté uloženy instance příslušných komponent. V případě \textit{arch type} z tabulky~\ref{tab:arch-type} by tento \textit{arch type} obsahoval jedno pole pro \texttt{Position} komponenty, jedno pole pro \texttt{Health} komponenty a jedno pole pro \texttt{Damage} komponenty.

Pokud bychom chtěli iterovat přes všechny entity s danými komponentami, stačilo by nám projít příslušná pole všech \textit{arch type} s těmito komponentami. Tato iterace by byla, díky sekvenčnímu přístupu zmíněnému v minulé sekci, velice rychlá. Ke \textit{cache miss} by docházelo pouze při přechodu na další \textit{arch type}.

Nevýhodou knihoven založených na \textit{arch type} bývá pomalé přidávání a odebírání komponent. Pokud je entitě přidána nebo odebrána komponenta, dojde ke změně jejího \textit{arch type}. Při této změně se vezmou instance všech komponent dané entity a přesunou se do nového \textit{arch type}.

Pro více informací o \textit{arch type} a o tom jak je ECS knihovny využívají je možné nahlédnout do série článků \textit{ECS back and forth}~\cite{Caini_2019} od Michela Cainiho.

\subsection{Stanovení hypotézy}
\label{hypothesis}
Nyní stanovíme hypotézu. Jednotlivé ECS knihovny lze rozdělit do kategorií na základě jistých vlastností. My tyto kategorie popíšeme a stanovíme jaké bude jejich pořadí při výkonnostním porovnání.

Nejrychleji vyjdou ECS knihovny, které vyžadují aby byli komponenty reprezentovany jako struktury a zároveň používají \textit{arch type}. Jak již bylo zmíněno, díky sekvenčnímu přístupu, který \textit{arch type} využívá, dojde k minimalizaci počtu \textit{cache miss}. To má za následek vysoký výkon. Z toho důvodu tyto knihovny vyjdou nejlépe. Označme tuto kategorii jako kategorii 1.

Některé ECS knihovny vyžadují, aby jednotlivé komponenty, byli reprezentovány jako třídy. Tyto knihovny vyjdou nejhůře. V případě, že komponenta je třída, znamená to, že její proměnné musejí být pointery. Z toho důvodu iterace přes komponenty v těchto knihovnách vede ve velmi velký počet \textit{cache miss}, kvůli kterému se tyto knihovny umístí nejhůře. Označme tuto kategorii jako kategorii~3.

Zbývají knihovny, které používají \textit{arch type} a zároveň vyžadují aby komponenty byli reprezentovány jako třídy. Tyto knihovny se umístí hůře než knihovny z kategorie 1, ale lépe než knihovny z kategorie 3. Označme tuto kategorii jako kategorii 2.

\section{Měřené knihovny}
V této sekci se budeme zabývat ECS knihovnami, které budeme měřit. První si uvedeme odkud měřené ECS knihovny bereme a poté si jednotlivé ECS knihovny krátce představíme.

V sekci~\ref{sec:ecs-libs} jsme zmínili, že existuje repositář \textit{EcsCsharpBenchmark}~\cite{EcsCsharpBenchmark} na platformě GitHub obsahující výkonnostní porovnání ECS knihoven pro C\#. Tento repositář byl inspirací této práce, která také porovnává jednotlivé ECS knihovny, ale na místo jednoduchých testů je porovnává na hře. V našem měření budeme měřit knihovny, které byli porovnávány ve zmiňovaném repositáři \textit{EcsCsharpBenchmark}~\cite{EcsCsharpBenchmark}.

Od zahájené této práce byl repositář \textit{EcsCsharpBenchmark}~\cite{EcsCsharpBenchmark} několikrát upraven. Během úprav byli přidávány a odebírány některé ECS knihovny z porovnání. V této práci se zaměříme na knihovny, které byli v repositáři \textit{EcsCsharpBenchmark}~\cite{EcsCsharpBenchmark} porovnávány během commitu \textit{db67d1d}~\cite{EcsCsharpBenchmarkCommit}.

Nyní si představíme jednotlivé ECS knihovny, které budeme měřit. U každého si uvedeme přeložený popisek, který je možné nalézt v repositáři dané knihovny. Tyto popisky jsou doslovně převzaté a autor práce neručí za korektnost informací, které jsou jejich obsahem.

\begin{enumerate}
    \item \textbf{Arch~\cite{Arch}:} Vysoko-výkonnostní ECS knihovna založená na Archtype a Chunks určená pro herní vývoj a data-oriented programování.
    \item \textbf{DefaultEcs~\cite{DefaultEcs}:} ECS framework, který si klade za cíl být přístupný s minimálními omezeními, zatímco zachovává co největší výkon pro vývoj her.
    \item \textbf{Entitas~\cite{Entitas}:} Nejpopulárnější open-source ECS framework pro C\# a Unity.
    \item \textbf{HypEcs~\cite{HypEcs}:} Lightweight a snadno použitelná ECS knihovna s efektivní sadou funkcionalit pro tvorbu her.
    \item \textbf{LeoECS~\cite{LeoECS}:} Lightweight ECS framework pro C\#. Mezi hlavní cíle tohoto frameworku patří výkon, nulová nebo minimální alokace, minimalizace využití paměti a absence závislostí na jakémkoliv herním enginu. (Přeloženo pomocí ChatGPT.)
    \item \textbf{LeoEcsLite~\cite{LeoEcsLite}:} Lightweight ECS framework pro C\#. Mezi hlavní cíle tohoto frameworku patří výkon, nulová nebo minimální alokace, minimalizace využití paměti a absence závislostí na jakémkoliv herním enginu. (Přeloženo pomocí ChatGPT.)
    \item \textbf{MonoGameExtended.Entities~\cite{MonoGameExtended}:} Moderní vysoce výkonnostní ECS framework založený na Artemis.
    \item \textbf{RelEcs~\cite{RelEcs}:} Lightweight a snadno použitelná ECS knihovna s efektivní sadou funkcionalit pro tvorbu her.
    \item \textbf{Svelto.ECS~\cite{SveltoECS}:} Reálný ECS framework pro C\#. Umožňuje psát zapouzdřený, oddělený, udržovatelný, vysoce efektivní, datově orientovaný, cache-přátelský kód bez bolesti.
\end{enumerate}

V minulé sekci jsme si definovali kategorie do kterých jednotlivé ECS knihovny spadají. Do kategorie 1 patří \textit{Arch}, \textit{HypEcs} a \textit{Svelto.ECS}. Do této kategorie také zařadíme \textit{DefaultEcs}, \textit{LeoECS} a \textit{LeoEcsLite}, které sice \textit{arch type} nepoužívají, ale používají datovou strukturu, která také využívá sekvenčního přístupu pro minimalizaci počtu \textit{cache miss}. V kategorii 2 je pouze \textit{RelEcs}. Do kategorie 3 spadají \textit{MonoGameExtended.Entities} a \textit{Entitas}.

\section{Výsledky}
\label{sec:benchmark-results}
Analýzu měření jsme provedli v sekci~\ref{benchmark-analysis}. Poté v sekci~\ref{benchmark-implementation} jsme si rozebrali implementaci měření. V této sekci se budeme věnovat výsledkům tohoto měření.

Předtím něž si prezentujeme výsledky měření, tak si shrňme jak jej provádíme. Prvně si vytvoříme novou instanci hry (instanci třídy \texttt{Game}) s danou ECS knihovnou (s danou instancí třídy dědicí od \texttt{ECSFactory}). Poté odsimulujeme přípravný počet iterací (provedeme \texttt{setupIterationCount} iterací \textit{game loop} naší hry). Následně odsimulujeme měřený počet iterací (\texttt{benchmarkIterationCount}) a přitom budeme měřit čas. To několikrát zopakujeme (přesněji námi zvolený framework to za nás několikrát zopakuje) a získáme průměrný čas jak rychle hra zvládne odsimulovat měřený počet iterací (\texttt{benchmarkIterationCount}). Test provádíme pro každou měřenou ECS knihovny (ty jsme si rozebrali v minulé sekci).

Měření budeme spouštět vícekrát s různými velikostmi herního světa. Velikost herního světa je definována v \texttt{GameWorld.Size}. Tento statický field obsahuje výšku a šířku herního světa. V sekci~\ref{benchmark-implementation} jsme si popsali jednotlivé parametry našeho měření, následující tabulka udává jejich nastavení:

\begin{table}[!htb]
    \centering\footnotesize\sf
    \begin{tabular}{c c c c}
        \toprule
        název parametru & nastavená hodnota \\
        \midrule
        \texttt{seed} & 0 \\
        \texttt{deltaTime} & 1 / 60 \\
        \texttt{setupIterationCount} & 30 * 60 \\
        \texttt{benchmarkIterationCount} & 60 * 60 \\
        \bottomrule
    \end{tabular}
    \caption{Nastavení parametrů našeho měření.}
    \label{tab:benchmark-parameters}
\end{table}

Parametr \texttt{deltaTime} je nastaven na 1/60s a to odpovídá 60 FPS (snímkům za sekundu). Poté parametr \texttt{setupIterationCount} odpovídá zhruba 30 sekundám simulace a parametr \texttt{benchmarkIterationCount} je přibližně 1 minuta simulace. Měření byla prováděna po počítači s procesorem \xxx{X}, grafickou kartou \xxx{Y} a 16GB RAM.

\subsection{První měření}
První měření bylo spuštěno s velikostí světa 8192x8192. Výsledky tohoto měření:

\begin{table}[!htb]
    \centering\footnotesize\sf
    \begin{tabular}{c c c c}
        \toprule
        knihovna & Průměrný čas & Chyba & StdDev \\
        \midrule
        HypEcs & 1.133 s & 0.0218 s & 0.0194 s \\
        LeoEcsLite & 1.194 s & 0.0188 s & 0.0176 s \\
        LeoECS & 1.208 s & 0.0233 s & 0.0229 s \\
        DefaultEcs & 1.216 s & 0.0243 s & 0.0451 s \\
        Svelto.ECS & 1.384 s & 0.0172 s & 0.0161 s \\
        Arch & 1.394 s & 0.0098 s & 0.0092 s \\
        RelEcs & 1.401 s & 0.0166 s & 0.0156 s \\
        MonoGameExtended.Entities & 2.004 s & 0.0340 s & 0.0318 s \\
        Entitas & 4.732 s & 0.0244 s & 0.0228 s \\
        \bottomrule
    \end{tabular}
    \caption{Výsledky prvního měření pro jednotlivé ECS knihovny.}
    \label{tab:first-benchmark-results}
\end{table}

Tabulka zachycuje výsledné časy prvního měření. V první sloupci jsou názvy jednotlivých ECS knihoven. Ve druhém sloupci jsou naměřené časy. Ve třetím sloupci je možná chyba a ve čtvrtém standardní odchylka. Jednotlivé řádky jsou seřazené podle naměřených časů.

Na obrázku~\ref{tab:first-benchmark-results} je možné vidět graf znázorňující výsledky z tabulky~\ref{tab:first-benchmark-results}. Tento graf je obarvený podle kategorií, které jsme si stanovili v sekci~\ref{hypothesis}. Knihovny, které spadají do kategorie 1, jsou obarveny zeleně, následně knihovny, které patří do kategorie 2, jsou obarveny žlutě a knihovny které, jsou součástí kategorie 3, jsou obarveny červeně.

\begin{figure}[!htb]
    \centering
    \includegraphics[width=1.0\linewidth]{plots/first_benchmark_results.pdf}
    \caption{Grafy vyobrazující výsledky prvního měření zachycené v tabulce~\ref{tab:first-benchmark-results}. Graf je obarvený podle kategorií ze sekce~\ref{hypothesis}.}
    \label{fig:first-benchmark-results}
\end{figure}

V sekci~\ref{hypothesis} jsme si stanovili hypotézu, podle které mají nejlépe vyjít knihovny z kategorie 1 (viz zelená barva). Poté se mají umístit knihovny z kategorie 2 (viz žlutá barva). A na závěr se mají umístit knihovny z kategorie 3 (viz červená barva). Při nahlédnutí do tabulky~\ref{tab:first-benchmark-results} nebo grafu z obrázku~\ref{fig:first-benchmark-results} je lehké pozorovat, že tomu tak opravdu je. Ovšem je nutné podotknout, že knihovny \textit{Arch} a \textit{RelEcs} mají velmi podobný výsledek a pokud nahlédneme na chybu z tabulky~\ref{tab:first-benchmark-results} tak je možné pozorovat, že časy jsou téměř identické. Vyplývá z toho to, že pro tento počet entit požadavek na to, aby jednotlivé komponenty byli struktury, nemusí nutně vést k velkému výkonnostnímu zlepšení.

\newpage

\subsection{Druhé měření}
Druhé měření bylo spouštěno s velikostí světa 16384x16384. Výsledky tohoto měření:

\begin{table}[!htb]
    \centering\footnotesize\sf
    \begin{tabular}{c c c c}
        \toprule
        knihovna & Průměrný čas & Chyba & StdDev \\
        \midrule
        HypEcs & 7.789 s & 0.1542 s & 0.3044 s \\
        DefaultEcs & 7.958 s & 0.1585 s & 0.3129 s \\
        LeoECS & 8.234 s & 0.1647 s & 0.3011 s \\
        LeoEcsLite & 8.471 s & 0.1670 s & 0.3177 s \\
        Svelto.ECS & 8.589 s & 0.1711 s & 0.3571 s \\
        Arch & 8.938 s & 0.1775 s & 0.4253 s \\
        RelEcs & 13.862 s & 0.2707 s & 0.3424 s \\
        MonoGameExtended.Entities & 20.442 s & 0.2375 s & 0.2222 s \\
        Entitas & 25.908 s & 0.5035 s & 0.6183 s \\
        \bottomrule
    \end{tabular}
    \caption{Výsledky druhého měření pro jednotlivé ECS knihovny.}
    \label{tab:second-benchmark-results}
\end{table}

\begin{figure}[!htb]
    \centering
    \includegraphics[width=1.0\linewidth]{plots/second_benchmark_results.pdf}
    \caption{Grafy vyobrazující výsledky druhého měření zachycené v tabulce~\ref{tab:second-benchmark-results}. Graf je obarvený podle kategorií ze sekce~\ref{hypothesis}.}
    \label{fig:second-benchmark-results}
\end{figure}

Z tabulky~\ref{tab:second-benchmark-results} i grafu~\ref{fig:second-benchmark-results} je možné vypozorovat významnější rozdíl mezi knihovnami první kategorie (zelená barva) a knihovnou druhé kategorie (žlutá barva). Za tento rozdíl mohou převážně \textit{cache misses} způsobené tím, že knihovna \textit{RelEcs} používá třídy namísto struktur pro reprezentaci komponent (viz sekce~\ref{hypothesis}). Lze pozorovat, že hypotéza zde byla také splněna.

\subsection{Třetí měření}
Třetí měření bylo spouštěno s velikostí světa 32768x32768. Výsledky tohoto měření:

\begin{table}[!htb]
    \centering\footnotesize\sf
    \begin{tabular}{c c c c}
        \toprule
        knihovna & Průměrný čas & Chyba & StdDev \\
        \midrule
        DefaultEcs & 196.2 s & 3.91 s & 6.63 s \\
        LeoECS & 208.6 s & 3.32 s & 3.10 s \\
        HypEcs & 210.3 s & 2.05 s & 1.82 s \\
        LeoEcsLite & 218.6 s & 3.08 s & 2.88 s \\
        Arch & 220.0 s & 4.27 s & 3.99 s \\
        Svelto.ECS & 229.5 s & 4.52 s & 5.38 s \\
        RelEcs & 236.7 s & 3.60  s & 3.37 s \\
        Entitas & 308.6 s &6.11 s & 6.01 s \\
        MonoGameExtended.Entities & 360.9 s & 1.86 s & 1.74 s \\
        \bottomrule
    \end{tabular}
    \caption{Výsledky třetího měření pro jednotlivé ECS knihovny.}
    \label{tab:third-benchmark-results}
\end{table}

\begin{figure}[!htb]
    \centering
    \includegraphics[width=1.0\linewidth]{plots/third_benchmark_results.pdf}
    \caption{Grafy vyobrazující výsledky třetího měření zachycené v tabulce~\ref{tab:third-benchmark-results}. Graf je obarvený podle kategorií ze sekce~\ref{hypothesis}.}
    \label{fig:third-benchmark-results}
\end{figure}

Z tabulky~\ref{tab:third-benchmark-results} i grafu~\ref{fig:third-benchmark-results} lze pozorovat, že výsledky jednotlivých kategorií jsou opět, podobně jako v případě prvního měření, blíže u sebe. Může za to pravděpodobně to, že hra momentální tráví spoustu času ve složitých datových strukturách jako jsou \textit{behavior stromy} a \textit{kd-stromy}, tím pádem knihovny první kategorie (zelená barva) nezískávají tolik výkonu na dobrém využití \textit{cache}. Je možné pozorovat, že hypotéza je stále splněna.

\newpage

\subsection{Zajímavé knihovny}
Zajímavou knihovnou je knihovna \textit{Svelto.ECS}, která vyžaduje, aby bylo během kompilace známo, z kterých komponent se jednotlivé entity budou skládat. Je lehké pozorovat že tato omezení nevedla k většímu výkonnostnímu zlepšení oproti ostatním knihovnám z této kategorie.

Knihovny \textit{HypEcs} a \textit{RelEcs} jsou od stejného autora. Autor těchto knihoven udělal nejprve \textit{RelEcs} a poté \textit{HypEcs}. \textit{HypEcs} se mu podařilo udělat rychlejší zejména, protože použil struktury místo třídy pro reprezentaci komponent.

Knihovny \textit{LeoECS} a \textit{LeoEcsLite} jsou od stejného autora. Obě tyto knihovny mají podobný výkon, ovšem při nahlédnutí do výsledků měření projektu \textit{Ecs.Csharp.benchmark}~\ref{EcsCsharpBenchmark} (konkrétně commitu \textit{db67d1d}~\cite{EcsCsharpBenchmarkCommit}), je možné pozorovat, že tyto dvě knihovny se umístili velmi špatně. Hlavním důvodem je to, že autor projektu \textit{Ecs.Csharp.benchmark} použil Nugget balíčky pro tyto knihovny, které byli vytvořeny jiným autorem. Autor projektu \textit{Ecs.Csharp.benchmark} dokonce tvrdí, že tyto balíčky byli sestaveny v debug konfiguraci.

ECS knihovna \textit{Entitas} je na platformě GitHub~\cite{GitHub} nejpopulárnější (má nejvíce hvězdiček) ECS knihovnou pro C\#. I přesto, že v našem měření vyšla nejhůře, její výkon je pořád dost dobrý. K její popularitě také přispívá větší podpora pro herní engine \textit{Unity}~\cite{Unity}.