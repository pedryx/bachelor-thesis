\chapter{Popis Implementace}

\section{Páteřní kód}

\section{Abstrakční vrstva}

\section{Herní svět}

\section{Herní data}

\section{Umělá inteligence???}

\section{Vesnice a vesničané}

\section{Systémy a komponenty}

% \chapter{Popis implementace}
% V této kapitole si blíže představíme implementaci naší hry.

% \section{Abstrakce}
% Prvně si rozebereme jak vypadá páteřní kód naší hry.

% % Game, GameState, LevelState, Factory, LevelFactory

% Nyní si přibližme naší abstrakční vrstvu Abstract.ECS.

% % \xxx{Code base}
% % \\
% % \xxx{Abstract.ECS layer}

% \section{Herní svět}
% Herní svět je reprezentován třídou GameWorld.

% O generování herního světa se stará třída GameWorldGenerator.

% Rozeberme si co jsou to vlastně shadery.

% Naše hra používá dva shadery.

% % \xxx{GameWorld}
% % \\
% % \xxx{GameWorldGenerator}
% % \\
% % \xxx{Shaders}

% \section{Herní data a jejich reprezentace}

% Pro správu herních dat, slouží třída GlobalInstances.

% % co jsou herni data, co trida dela

% % \xxx{Data - biomes, items, and resources}

% \section{Systémy a komponenty}

% Následující seznam obsahuje a popisuje všechny herní systémy a komponenty.