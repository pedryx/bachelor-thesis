\chapter{Popis Implementace}
V této kapitole si představíme klíčové a zajímavé části implementace.

\section{Páteřní kód}
Páteřní kód naší hry se skládá z následujících tříd.

\begin{enumerate}
    \item \verb|Game|
    \item \verb|GameState|
    \item \verb|LevelState|
    \item \verb|Factory|
    \item \verb|LevelFactory|
\end{enumerate}

\section{Abstrakční vrstva???}

\section{Herní svět}
Herní svět je reprezentován třídou \verb|GameWorld|.

O jeho generování se stará třída \verb|GameWorldGenerator|.

Pro generování světa používáme 2 shadery.

Společná část obou shaderů.

Popis compute shaderu.

Popis fragment shaderu.

\section{Herní data a jejich reprezentace}
Popis co jsou herní data.

Pro jejich správu používáme |GlobalInstances|.

Popis třídy.

Herní data v naší hře.

\begin{enumerate}
    \item biomy
    \item předměty
    \item suroviny
\end{enumerate}

\section{Umělá inteligence???}

\section{Systémy a komponenty???}









% \chapter{Popis implementace}
% V této kapitole si blíže představíme implementaci naší hry.

% \section{Abstrakce}
% Prvně si rozebereme jak vypadá páteřní kód naší hry.

% % Game, GameState, LevelState, Factory, LevelFactory

% Nyní si přibližme naší abstrakční vrstvu Abstract.ECS.

% % \xxx{Code base}
% % \\
% % \xxx{Abstract.ECS layer}

% \section{Herní svět}
% Herní svět je reprezentován třídou GameWorld.

% O generování herního světa se stará třída GameWorldGenerator.

% Rozeberme si co jsou to vlastně shadery.

% Naše hra používá dva shadery.

% % \xxx{GameWorld}
% % \\
% % \xxx{GameWorldGenerator}
% % \\
% % \xxx{Shaders}

% \section{Herní data a jejich reprezentace}

% Pro správu herních dat, slouží třída GlobalInstances.

% % co jsou herni data, co trida dela

% % \xxx{Data - biomes, items, and resources}

% \section{Systémy a komponenty}

% Následující seznam obsahuje a popisuje všechny herní systémy a komponenty.