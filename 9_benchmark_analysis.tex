\chapter{Analýza měření}
V této kapitole si rozebereme jaké existují přístupy k měření a na co při měření dbát ohled.

\section{Problematika měření}
Měření výkonu je komplexní proces, který by mohl být samostatnou prací sám o sobě.

Benchmark.NET je populární framework pro platformu .NET, který lze použít k měření výkonu.

\section{Přístupy k měření}
Nejprve je nutné si říct co to je vlastně game loop.

Nejpřímočarejší způsob je změřit jak dlouho trvá jedna iterace.

Lepší způsob by tedy bylo změřit n iterací a spočítat jejich průměr.

Rendering může být časově naročnější proces, rozeberme si jestli by měl či neměl být součastí měření.

Dalším způsobem jak měřit výkon, je změřit iterace jednotlivých systémů.
% proc to nedelame

Výkon je také možné měřit během hraní samotné hry.
% proc to nedelame

% \xxx{Delta time jedne iterace}
% \\
% \xxx{Delta time n iteracich}
% \\
% \xxx{Mel by byt rendering soucasti mereni}
% \\
% \xxx{Mereni prumerneho delta time jednotlivych ssytemu}
% \\
% \xxx{Mereni behem hrani}

\section{Na co při měření neberem ohled?}
Některé ECS knihovny pro získání lepšího výkonu používají paralelní queries.
% co to je
% nase hra je nepouziva protoze je z jednoduchosti navrzena single threaded

Další častá optimalizace, co některé ECS knihovny nabízejí jsou reaction systémy.
% co to je
% nepouzivame to protoze to nemaji vsechny knihovny

Jednotlivé ECS knihovny často používají další specifické výkonostní funkcionality.
% mi je nepouzivame protoze je nemaji vsechny knihovny

Další věc na kterou při měření neberem ohledy je přídávání a odebírání komponent.
% sparse set muzou by zvyhodnene
% nas herni design je prizpusoben tomu, ze to neovlivni mereni

% \xxx{Paralelni queries}
% \\
% \xxx{Reaction systemy}
% \\
% \xxx{Pridavani/odebirani komponent za runtime - design prizpusoben}
% \\
% \xxx{jine vykonostni vylepseni specificke pro danou ECS knihovnu}