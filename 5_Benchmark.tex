\chapter{Měření}
Náš projekt se skládá ze dvou částí a to hry a poté měření, které využívá vytvořenou hru pro měření výkonu jednotlivých ECS knihoven. Analýzu tohoto měření jsme provedli v sekci~\ref{benchmark-analysis} a jeho implementaci jsme popsali v sekci~\ref{benchmark-implementation}. V této kapitole si nejprve stanovíme hypotézu, poté si přiblížíme ECS knihovny, které budeme měřit a na závěr si prezentujeme výsledky tohoto měření.

\section{Hypotéza}
\label{chap:hypothesis}
V této sekci si stanovíme hypotézu našeho měření. V našem měření měříme jak dlouho trvá odsimulovat určitý počet iterací naší hry s jednotlivými ECS knihovnami. Ty se dají rozdělit na kategorie podle určitých vlastností. V této kapitole stanovíme jak si myslíme, že se jednotlivé kategorie umístí.

% Do teď jsme se věnovali samotné hře, kterou v následujících kapitolách budeme chtít použít pro porovnání výkonu ECS knihoven. Ale ještě předtím, konkrétně v této kapitole, si rozebereme předpokládané výsledky tohoto porovnání.

\subsection{Cache}
Využití cache je klíčový faktor pro výkon ECS knihovny, proto si nyní připomene co to vlastně je a jak funguje.

Program si svoje data uchovává v paměti RAM. Ovšem přístup do této paměti je z hlediska času drahý. Pro představu u moderních pamětí takovíto přístup může trvat vyšší desítky nanosekund. Z tohoto důvodu se využívají menší paměti, takzvané cache, které disponují mnohem vyšší rychlostí ale mnohem menší velikostí. Pro představu moderní procesory používají cache, u kterých přístup může trvat menší desetiny nanosekund.

Při přístupu k datům se procesor nejprve podívá zda nemá data již v této cache. Pokud ano, jedná se o takzvaný \textit{cache hit} a data si z ní načte. Pokud ne, jedná se o takzvaný \textit{cache miss} a je nutné data načíst z hlavní paměti. 

Již víme, že \textit{cache miss} jsou drahé, proto je potřeba pro vysoký výkon je minimalizovat. Ale jak přesně procesor rozhoduje o tom, jaká data budou v cache a jaká ne. Využívá se takzvané \textit{časové} a \textit{prostorové lokality}. \textit{Časová lokalita} spočívá v tom, že pokud jsme přistoupili k nějakým datům, je velká šance, že k nim brzo budeme chtít přistoupit znovu. Nadruhou stranu \textit{prostorová lokalita} spočívá v tom, že pokud jsme přistoupili k nějakým datům, je velká šance, že budeme chtít přistoupit také k datům, které se nacházejí blízko nich.

Jedna z možností, jak minimalizovat počet \textit{cache miss}, je využít sekvenčního přístupu. Pokud budeme přistupovat k datům, které jsou v paměti hned za sebou, tak díky \textit{prostorové lokalitě} bude počet \textit{cache miss} velmi malí.

Je nutné upozornit, že výše popsaný model cache je úmyslně zjednodušen. Velice detailní popis cache lze nalézt v \textit{Cache Memories}~\cite{10.1145/356887.356892} od \textit{Alana Jaye Smitha}. Stručnější popis a také informace o tom jak lze cache využít lze nalézt v již zmiňované knize \textit{Game Programming Patterns}~\cite{nystrom2014game}, konkrétně v kapitole \textit{Data Locality}. 

\subsection{Arch type}
Některé ECS knihovny pro lepší výkon používají \textit{arch type}. Jedná se o datovou strukturu pro ukládání komponent. Knihovny, které používají tuto datovou strukturu mají většinou velký výkon, proto před stanovením hypotézy si použití \textit{arch type} lehce přiblížíme.

Jak již bylo zmíněno, \textit{arch type} je datová struktura. Tato datová struktura představuje typ entity. Tento typ je jednoznačně určen typy všech komponent dané entity. Je možné si jej představit jako tabulku, kde jednotlivé sloupečky odpovídají komponentám a v každém řádku se nacházejí instance komponent pro danou entitu. Například můžeme mít \textit{arch type} vyobrazený v tabulce~\ref{tab:arch-type}. Tento \textit{arch type} je jednoznačně určen typy komponent \texttt{Position} (představující pozici entity), \texttt{Health} (představující počet životů entity) a \texttt{Damage} (představující poškození, které entita může udělit jiné entitě).

\begin{table}[!htb]
    \centering\footnotesize\sf
    \begin{tabular}{c c c c}
        \toprule
        entita & \texttt{Position} & \texttt{Health} & \texttt{Damage} \\
        \midrule
        hráč & $(0,0)$ & 10 & 3 \\
        nepřítel & (5,4) & 6 & 1 \\
        npc & (-4,8) & 16 & 9\\
        \bottomrule
    \end{tabular}
    \caption{Tabulka vyobrazující \textit{arch type}, který je jednoznačně určen komponentami \texttt{Position}, \texttt{Health} a \texttt{Damage}. Tento \textit{arch type} obsahuje tři entity, konkrétně hráče, nepřítele a npc.}
    \label{tab:arch-type}
\end{table}

Jednotlivé \textit{arch type} jsou uloženy ve \textit{world}. Každý \textit{arch type} má v sobě několik polí, konkrétně jedno pole pro každý typ komponenty. V každém poli jsou poté uloženy instance příslušných komponent. V případě \textit{arch type} z tabulky~\ref{tab:arch-type} by tento \textit{arch type} obsahoval jedno pole pro \texttt{Position} komponenty, jedno pole pro \texttt{Health} komponenty a jedno pole pro \texttt{Damage} komponenty.

Pokud bychom chtěli iterovat přes všechny entity s danými komponentami, stačilo by nám projít příslušná pole všech \textit{arch type} s těmito komponentami. Tato iterace by byla, díky sekvenčnímu přístupu zmíněnému v minulé sekci, velice rychlá. Ke \textit{cache miss} by docházelo pouze při přechodu na další \textit{arch type}.

Nevýhodou knihoven založených na \textit{arch type} bývá pomalé přidávání a odebírání komponent. Pokud je entitě přidána nebo odebrána komponenta, dojde ke změně jejího \textit{arch type}. Při této změně se vezmou instance všech komponent dané entity a přesunou se do nového \textit{arch type}.

Pro více informací o \textit{arch type} a o tom jak je ECS knihovny využívají je možné nahlédnout do série článků \textit{ECS back and forth}~\cite{Caini_2019} od Michela Cainiho.

\subsection{Hypotéza}
Nyní vyřkneme hypotézu. Jednotlivé ECS knihovny lze rozdělit do kategorií na základě jistých vlastností. My tyto kategorie popíšeme a stanovíme jaké bude jejich pořadí při výkonnostním porovnání.

Nejrychleji vyjdou ECS knihovny, které používají \textit{arch type}. Jak již bylo zmíněno, \textit{arch type} je datová struktura s kterou lze, díky malému počtu \textit{cache miss}, dosáhnout velkého výkonu při iteraci entit. Z těchto důvodů ECS knihovny, které \textit{arch type} používají vyjdou nejlépe.

Některé ECS knihovny vyžadují aby jednotlivé komponenty byli reprezentovány jako třídy. Tyto knihovny vyjdou nejhůře. V případě, že komponenta je třída, znamená to, že její proměnné musejí být pointery. Z toho důvodu iterace přes komponenty v těchto knihovnách vede ve velmi velký počet \textit{cache miss}, kvůli kterému se tyto knihovny umístí nejhůře. 

Zbývají knihovny, které povolují komponenty reprezentovat jako struktury, ale zároveň nepoužívají \textit{arch type}. Tyto knihovny se umístí hůře než knihovny, které používají \textit{arch type}, ale lépe než knihovny, které vyžadují aby komponenty byli třídy.

\section{Měřené knihovny}
V této sekci se budeme zabývat ECS knihovnami, které budeme měřit. První si uvedeme odkud měřené ECS knihovny bereme a poté si jednotlivé ECS knihovny krátce představíme.

V sekci~\ref{sec:ecs-libs} jsme zmínili, že existuje repositář \textit{EcsCsharpBenchmark}~\cite{EcsCsharpBenchmark} na platformě GitHub obsahující výkonnostní porovnání ECS knihoven pro C\#. Tento repositář byl inspirací této práce, která také porovnává jednotlivé ECS knihovny, ale na místo jednoduchých testů je porovnává na hře. V našem měření budeme měřit knihovny, které byli porovnávány ve zmiňovaném repositáři \textit{EcsCsharpBenchmark}~\cite{EcsCsharpBenchmark}.

Od zahájené této práce byl repositář \textit{EcsCsharpBenchmark}~\cite{EcsCsharpBenchmark} několikrát upraven. Během úprav byli přidávány a odebírány některé ECS knihovny z porovnání. V této práci se zaměříme na knihovny, které byli v repositáři \textit{EcsCsharpBenchmark}~\cite{EcsCsharpBenchmark} porovnávány během commitu \textit{db67d1d}~\cite{EcsCsharpBenchmarkCommit}.

Nyní si představíme jednotlivé ECS knihovny, které budeme měřit. U každého si uvedeme přeložený popisek, který je možné nalézt v repositáři dané knihovny. Autor práce neručí za korektnost těchto popisků. Mezi měřené ECS knihovny patří:

\begin{enumerate}
    \item \textbf{Arch~\cite{Arch}:} Vysoko-výkonnostní ECS knihovna založená na Archtype určená pro herní vývoj a data-oriented programování.
    \item \textbf{DefaultEcs~\cite{DefaultEcs}:} ECS framework, který si klade za cíl být přístupný s minimálními omezeními, zatímco zachovává co největší výkon pro vývoj her.
    \item \textbf{Entitas~\cite{Entitas}:} Nejpopulárnější open-source ECS framework pro C\# a Unity.
    \item \textbf{HypEcs~\cite{HypEcs}:} Lightweight a snadno použitelná ECS knihovna s efektivní sadou funkcionalit pro tvorbu her.
    \item \textbf{LeoECS~\cite{LeoECS}:} Lightweight ECS framework pro C\#. Mezi hlavní cíle tohoto frameworku patří výkon, nulová nebo minimální alokace, minimalizace využití paměti a absence závislostí na jakémkoliv herním enginu. (Přeloženo přes ChatGPT.)
    \item \textbf{LeoEcsLite~\cite{LeoEcsLite}:} Lightweight ECS framework pro C\#. Mezi hlavní cíle tohoto frameworku patří výkon, nulová nebo minimální alokace, minimalizace využití paměti a absence závislostí na jakémkoliv herním enginu. (Přeloženo přes ChatGPT.)
    \item \textbf{MonoGameExtended.Entities~\cite{MonoGameExtended}:} Moderní vysoce výkonnostní ECS framework založený na Artemis.
    \item \textbf{RelEcs~\cite{RelEcs}:} Lightweight a snadno použitelná ECS knihovna s efektivní sadou funkcionalit pro tvorbu her.
    \item \xxx{Svelto.ECS?}
\end{enumerate}

% Stejně jako v repositáři \textit{EcsCsharpBenchmark}~\cite{EcsCsharpBenchmark} byli pro ECS knihovny \textit{LeoECS} a \textit{LeoEcsLite} použity nuget balíčky. Tyto nuget balíčky byli vytvořeny jiným autorem a byli zkompilované v debug režimu.

\section{Výsledky}
kontext

prezentace vysledku

oduvodneni vysledku