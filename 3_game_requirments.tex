\chapter{Co požadujeme od hry}
V této kapitole si rozebereme požadavky, které požadujeme od naší hry. Prvně si rozeberem jednotlivé charakteristiky, které by naše hra měla splňovat. Dále na základě těchto charakteristik zvolíme žánr naší hry a odlvodníme proč jsme jej zvolili.

% Než budeme pokračovat, pripomeňmě si prvně, že naší hru chceme použít pro měření výkonu ECS knihoven.

\section{Charakteristiky hry}
Aby nám naše hra napomohla kde splněním cílů této práce, je nutné aby splňovala určité charakteristiky. V této sekci si jednotlivé charakteristiky představíme a odůvodníme proč jsou pro nás důležité.

První charakteristika, kterou od hry vyžadujeme, je velký počet entit. Tato charakteristika nám také implikuje netriviální velikost herního světa. Díky většímu počtu entit zajistíme, že jednotlivé systémy budou vykonávat netriviální množství práce, což nám usnadní měření výkonu jednotlivých ECS knihoven.

Dále budem chtít aby hra nebyla příliš jednoduchá. Jak bylo zmíněno v úvodu, chceme provést lepší výkonostní porovnanní a na to je potřeba se přiblížit ke složitosti větších her.

Zároveň, ale nechceme, aby naše hra byla až příliš složití. Velká složitost by mohla odvádět od problému který řešíme.

Poslední charakteristika je, že hra by neměla využívat častého přidávání a odebírání komponent. Pokud by toho hra hodně využívala, tak ECS knihovny založené na sparse setu by byli zvýhodněné.

% \\
% \xxx{Jaka charakteristiky by nase hra mela mit?}
% \\
% \xxx{- Velky pocet entit.}
% \\
% \xxx{- Hra musi byt dostatecne slozita aby vysledek mel prakticky smysl.}
% \\
% \xxx{- Hra nesmi byt prislis slozita aby slozitost neodvadela od problemu.}
% \\
% \xxx{- Hra by nemela vyuzivat casteho pridavani/odebirani komponent.}
% \\
% \xxx{-- Kdyby ano, nektere knihovny by byli zvyhodnene.}

\section{Simulace}
Žánrem naší hry bude simulace, ale řekněmě si nejprve co to vlastně je. Simulace je žánr video her, který se snaší simulovat agenty v jejich prostředí. U nás agenti budou vesničani a jejich prostředí bude příroda.

Proč jsme zvolili právě simulaci? Pro simulaci je jednodušší (oproti ostatním žánrům) vymyslet herní design, který má velký počet entit a netriviálně velký svět. Další velkou výhodou je malí vliv uživatele na průběh hry, to nám lépe provádět měření. Konkrétně v naší hře hráč bude spíše pozorovatel, který sleduje jak simulace probíhá.

% \\
% \xxx{Proc jsme zvolili simulaci?}
% \\
% \xxx{- Velky pocet entit.}
% \\
% \xxx{- Mali vliv uzivatele na prubeh hry.}