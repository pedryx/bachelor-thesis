\chapter{Co požadujeme od hry}
V této kapitole si rozebereme požadavky, které požadujeme od naší hry. Prvně si rozeberem jednotlivé charakteristiky, které by naše hra měla splňovat. Dále na základě těchto charakteristik zvolíme žánr naší hry a odůvodníme proč jsme jej zvolili.

Než budeme pokračovat, pripomeňmě si prvně, že naší hru chceme použít pro měření výkonu ECS knihoven. Budem tedy chtít aby byla nezávsilá na konkrétní knihovně. Konkrétní knihovnu bude možné zvolit před spuštěním hry.

\section{Charakteristiky hry}
Aby nám naše hra napomohla kde splnění práce, budem chtít aby splňovala následujicí charakteristiky:

\begin{enumerate}
    \item Nezávislost na konkrétní ECS knihovně.
    \item Hra by neměla využívat častého přidávání a odebírání komponent.
    \item Velký počet entit.
    \item Adekvátní složitost hry.
\end{enumerate}

Nézávlislost na konkrétní ECS knihovně pro nás bude důležitá, abychom hru mohli použít pro měření výkonu jednotlivých knihoven. Hra bude namísto konkrétní ECS knihovny pracovat pouze s abstrakční vrstvou. Bude pro nás žádoucí aby se výkon naší hry bežicí s konkretní ECS knihovnou, implementujucí abstrakční vrstvu, blížil k výkonu hry, která by používala pouze konkrétní ECS knihovnu bez abstrakční vrstvy.

Časté přidávání a odebírání komponent by vedlo ke zvýhodnění ECS knihoven založených na sparse set. Cílem této práce ale není porovnat hnihovny založené na sparse set oproti knihovnám založených na arch type. To jestli hra využívá častého přidávání a odebírání komponent je závislé čistě na implementaci. 

Ukažme si to na příkladu, konkrétně bychom chtěli reprezentovat efekt toho, že entita hladoví. Pro knihovnu založenou na sparse set by jeden z typických přístupů byl vytvořit Starvation komponentu a přidat jí na každou entitu, která hladoví. Pro knihovny založené na arch type by ale tento přístup vedl ke ztrátě na výkonu. Jedním ze způsobů jak by se dal problém vyřešit je, že by každá entita co může hladovět obsahovala Hunger komponentu, ve které by byl flag Starvation který by rozhodoval o tom jestli daná entita hladoví.

Velký počet entit nám usnadní měření výkonu naší hry. Větší počet entit nám zaručí, že systémy budou vykonávat netriviální množství práce, zároveň si tím vyzkoušíme velký výkon, kterým ECS disponuje. Velký počet entit nám také implikuje větší herní svět.

Bude pro nás důležité aby naše hra nebyla příliš jednoduchá. Budem se chtím vyhnout tomu, aby naše meření nebylo jenom dalším jednoduchým meřením výkonu. Ovšem taky si musíme dát pozor na to aby nebyla příliš složitá, jelikož to by odvádělo pozornost od problému, který řešíme.

% \\
% \xxx{Jaka charakteristiky by nase hra mela mit?}
% \\
% \xxx{- Velky pocet entit.}
% \\
% \xxx{- Hra musi byt dostatecne slozita aby vysledek mel prakticky smysl.}
% \\
% \xxx{- Hra nesmi byt prislis slozita aby slozitost neodvadela od problemu.}
% \\
% \xxx{- Hra by nemela vyuzivat casteho pridavani/odebirani komponent.}
% \\
% \xxx{-- Kdyby ano, nektere knihovny by byli zvyhodnene.}

\section{Simulace}
Žánrem naší hry bude simulace, ale řekněmě si nejprve co to vlastně je. Simulace je žánr video her, který se snaží simulovat agenty v jejich prostředí. U nás agenti budou vesničani a jejich prostředí bude příroda.

Proč jsme zvolili právě simulaci? Pro simulaci je jednodušší (oproti ostatním žánrům) vymyslet herní design, který má velký počet entit a netriviálně velký svět. Další velkou výhodou je malí vliv uživatele na průběh hry, to nám umožní lépe provádět měření. Konkrétně v naší hře hráč bude spíše pozorovatel, který sleduje jak simulace probíhá.

% \\
% \xxx{Proc jsme zvolili simulaci?}
% \\
% \xxx{- Velky pocet entit.}
% \\
% \xxx{- Mali vliv uzivatele na prubeh hry.}