% Proc jsem delali rozhodnuti kdyz vznikala ta implementace.
\chapter{Analýza implementace}
V této kapitole prozkoumáme jaké rozhodnutí byla udělána během implementace a odůvodníme jejich zvolení.

% Prvně nahlédneme na abstrakční vrstvu a poté na samotnou hru.

\section{Volba frameworku/enginu}
Jedním z cílů této práce je udělat hru pomocí ECS. V sekci~\ref{sec:ecs-libs} jsme si říkali, že danou hru budeme chtít použít pro měření výkonu ECS knihoven, které používají programovací jazyk C\#. Proto C\# také zvolíme jako programovací jazyk pro naši hru. 

Pro tvorbu naší hry bude nutné zvolit herní engine nebo framework. Jak jsme si říkali v sekci~\ref{sec:abstract-layer} naše hra bude namísto konkrétní ECS knihovny, pracovat pouze s abstrakční vrstvou. Budeme tedy chtít aby pro pro naši hru běžící s tímto enginem či frameworkem bylo možné specifikovat konkrétní implementaci abstrakční vrstvy, se kterou bude hra pracovat. Dále tuto hru budeme chtít použít pro vytvoření porovnání různých ECS knihoven. Budeme tedy požadovat aby ji bylo možné spouštět z jednoho měření s různými implementacemi abstrakční vrstvy, které budou odpovídat daným ECS knihovnám.

Herní enginy, jako je například Unity, nám schovávají low-level části hry, jako je například game loop. Také často přebírají kontrolu nad projektem. Mohlo by tedy být problematické naši hru spouštět nebo využívat její části z jiného kódu. To by nám mohlo dělat potíže v případě specifikace konkrétní implementace abstrakční vrstvy nebo při provádění našeho porovnání, ze kterého bychom chtěli naši hru spouštět s různými implementacemi abstrakční vrstvy.

Zaměřme se tedy na namísto herních enginů, na herní frameworky. Pokud nahlédneme na vydané hry na platformě Steam, tak nejpopulárnějším herním frameworkem pro C\# je Microsoft XNA. Bohuzel tento framework už není delší dobou firmou Microsoft, která je jeho autorem, podporován. Ovšem jelikož byl Microsoft XNA framework velice populární, po skončení jeho podpory vzniklo několik open source reimplementací, které v podpoře a vývoji tohoto herního frameworku pokračují. Mezi nejpopulárnější z nich patří FNA~\cite{FNA} a MonoGame~\cite{MonoGame}. Oba tyto frameworky jsou si syntaxí velice podobné, my pro implementaci naši hry zvolíme MonoGame z důvodů osobní preference autora.

V pozdější sekci~\ref{sec:terrain} provedeme analýzu generování terénu. Na základě této analýzy vybereme přístup, který bude vyžadovat compute shader. Ovšem nativní MonoGame nemá podporu pro compute shadery, tím pádem použití nativního MonoGame by nám volbu tohoto přístupu znemožnilo. Z toho důvodu použijeme fork~\ref{MonoGameCptMax} MonoGame frameworku, který tuto podporu nabízí.

% analyza abstrakce
\section{Abstrakční vrstvy}
\label{section:abstract-layer-analysis}
Abstrakční vrstva pro nás je důležitá, jelikož chceme aby naše hra byla nezávislá na konkrétní ECS knihovně. Program jako takový bude pracovat namísto ECS knihovny pouze s touto abstrakční vrstvou. Pro každou ECS knihovnu, kterou budeme měřit bude nutné vytvořit implementaci této vrstvy. K volbě konkrétní ECS knihovny dojde před spuštěním programu.

\subsection{Rozhraní ECS knihoven}
Pro návrh abstrakční vrstvy bude nejprve nutné rozebrat jak jsou jednotlivé prvky ECS reprezentovány při jejich implementaci. Krom základních členů, jako jsou \textit{entity}, \textit{komponenty} a \textit{systémy}, si představíme ještě \textit{world} a \textit{query}.

\begin{enumerate}
    \item \textbf{\textit{Component}:} \textit{Komponenty}, jako takové, bývají často reprezentovány třídou nebo strukturou. Jak již bylo zmíněno v úvodu, \textit{komponenty} obsahují pouze data, nikoliv žádnou logiku. Proto tyto třídy a struktury neobsahují žádné metody.

    \item \textbf{\textit{Entity}:} \textit{Entita} by, podobně jako v návrhovém vzoru Component, mohla být reprezentována jako kolekce svých \textit{komponent}. Ovšem to se v praxi nedělá, namísto toho se používá řešení, které vede k lepšímu výkonu. Většina ECS implementacích reprezentuje \textit{entity} pouze jako jednoduchý identifikátor a jednotlivé \textit{komponenty} jsou na \textit{entity} mapovány skrze \textit{world}.

    \item \textbf{\textit{World}:} \textit{World} je spravce \textit{entit} a \text{komponent}. Tento správce slouží jako kontejner pro všechny \textit{entity} a \textit{komponenty} v herním světě. Jeho rozhraní nabízí funkce, pomocí niž je možné vytvářet a mazat jednotlivé \textit{entity}. Často obsahuje také funkce pro přidávání a odebírání \textit{komponent} jednotlivým \textit{entitám}.

    Některé ECS implementace reprezentují \textit{entity} pomocí malé struktury, obsahující identifikátor dané \textit{entity} společně s referencí na \textit{world} do kterého patří. To umožňuje mít na této struktuře bohaté rozhraní s funkcemi pro přidávání a odebírání \textit{komponent}.

    \item \textbf{\textit{System}:} Reprezentace \textit{systémů} se v jednotlivých ECS implementacích dost liší. Některé vyžadují aby \textit{systém} byl třída dědicí od abstraktního předka, jiné zase volí opačný extrém a dovolují \textit{systém} reprezentovat téměř jakkoliv. Navzdory velkým odlišnostem se ve velkém počtu ECS implementací často objevuje objekt, který mohou používat jednotlivé systémy pro iterování entit s určitou množinou komponent.

    \item \textbf{\textit{Query}:} \textit{Query} je objekt, který lze použít pro iterování \textit{entit} s určitou množinou \textit{komponent}. Pro jeho vytvoření je většinou nutné poskytnou informaci, nad jakými typy \textit{komponent} bude \textit{query} pracovat. Je například možné mít \textit{query}, které pracuje nad \textit{komponentami} \verb|Movement| a \verb|Position|. Takové \textit{query} by pak bylo schopné iterovat nad všemi \textit{entitami} co mají \textit{komponenty} \verb|Movement| a \verb|Position|. Pomocí tohoto \textit{query} by bylo možné implementovat \verb|MovementSystem|, který by řídil logiku pohybu.
\end{enumerate}

Pro reprezentaci \textit{systémů} v naší abstrakční vrstvě bude nejprve nutné si přiblížit jak jednotlivé ECS knihovny reprezentují \textit{systémy}. Jak již bylo zmíněno, reprezentace \textit{systémů} se v jednotlivých ECS knihovnách dost liší. Významný rozdíl spočívá ve způsobech, jakými iterují \textit{entity}. Tyto způsoby se dají rozdělit do tří kategorií, kde každá kategorie obsahuje tři způsoby. Nyní si rozebereme jednotlivé kategorie a způsoby. Vždy prvně představíme pseudokód pro způsoby z této kategorie a poté danou kategorii popíšeme.

\subsubsection{1. kategorie}

\begin{enumerate}
    \item \verb|Query((entity) => { /**/ })|
    \item \verb|Query((component1, component2) => { /**/ })|
    \item \verb|Query((components1[], components2[]) => { /**/ })|
\end{enumerate}

\subsubsection{2. kategorie}

Způsoby z této kategorie mají většinou definovanou metodu \verb|Query| (neplést s \textit{query} jako objektem pro iterování \textit{entit}). Tato metoda bývá často implementována na \textit{query} nebo na \textit{world}. Této metodě se předává lambda funkce, která je v případě 1 volána na každou iterovanou \textit{entitu}, v případě 2 volána na množinu žádoucích instancí \textit{komponent} každé iterované \textit{entity}, v případě 3 volána na množinu polí, které obsahují žádoucí instance \textit{komponent} a kde jako index do těchto polí lze použít identifikátor příslušné \textit{entity}.

\begin{enumerate}
    \setcounter{enumi}{3}
    \item \verb|void Update(entity) { /**/ }|
    \item \verb|void Update(component1, component2) { /**/ }|
    \item \verb|void Update(components1[], components2[]) { /**/ }|
\end{enumerate}

\subsubsection{3. kategorie}

Způsoby z druhé kategorie vyžadují aby \textit{systém} byl třída, dědicí od rozhraní nebo předka. Toto rozhraní nebo předek nabízí abstraktní metodu \verb|Update|, kterou je nutné přetížit. Při iteraci je v případě 4 metoda \verb|Update| volána na každou iterovanou \textit{entitu}, v případě 5 na množinu žádoucích instancí \textit{komponent} každé iterované \textit{entity}, v případě 6 na množinu polí, které obsahují žádoucí instance \textit{komponent} a kde jako index do těchto polí lze použít identifikátor příslušné \textit{entity}.

\begin{enumerate}
    \setcounter{enumi}{6}
    \item \verb|foreach (entity in entities) { /**/ }|
    \item \verb|foreach ((component1, component2) in entities) { /**/ }|
    \item \verb|for (int i = 0; i < entitiesCount; i++)|\\\verb|{ components1[i], components2[i], /**/ }|
\end{enumerate}

Třetí kategorie nabízí kolekce nebo iterátory, které lze iterovat. Tyto kolekce nebo iterátory lze získat voláním příslušné funkce na \textit{world} nebo \textit{query}. V případě 7 tato kolekce nebo iterátor obsahuje všechny \textit{entity}, které chceme iterovat. V případě 8 tato kolekce nebo iterátor obsahuje n-tice žádoucích instancí \textit{komponent} každé iterované \textit{entity}. V případě 9 je namísto kolekce použita množina polí, které obsahují žádoucí instance \textit{komponent} a kde jako index do těchto polí lze použít identifikátor příslušné \textit{entity}.

\subsection{Inlinování funkce}
Při pohledu na způsoby pro iterování entit z minulé sekce přirozeně vyplývá pro každou entitu zavolat funkci (konkrétně na místě komentáře \verb|/**/|), která danou entitu zpracuje. Ovšem volání funkce nás stojí čas. Tento čas není moc velký a je většinou zanedbatelný s porovnáním s časem vykonávání funkce samotné. Ovšem my bychom chtěli volat funkci pro každou žádoucí entitu a v případě, že by tato funkce byla jednoduchá (její vykonání by trvalo malé množství času), tak se jednotlivé časy volání této funkce nasčítají a naše řešení by vedlo ke ztrátě na výkonu. Předtím než si navrhneme samotné systémy, bude nutné si tuto problematiku více přiblížit.

Některé ECS knihovny se snaží tomuto problému předejít tím, že namísto toho aby jejich systémy přijímali funkci, kterou zavolají na každou entitu, poskytnou uživateli kolekce nebo iterátory, které si uživatel sám zpracuje. Tím se volání funkce vyhnou úplně. Znamená to tedy, že náš způsob ublíží na výkonu pouze některým knihovnám a to by mělo vliv na naše porovnání. Konkrétně se jedná o způsoby 3, 6, 7, 8 a 9 z minulé sekce.

JIT používá optimalizaci, při které se nahradí místo volání funkce samotným obsahem dané funkce. Díky tomu může být volání takovéto funkce zadarmo. Konkrétně se jedná o takzvané \textit{inlinování funkce} (\textit{inline expansion}). Ovšem JIT tuto optimalizaci ne vždy provede, jelikož v některých případech se to nevyplatí nebo to dokonce udělat nemůže.

Pro více informací o tom kdy dochází a nedochází k \textit{inlinování funkcí}, je možné nahlédnout do článků od Davida Notaria~\cite{Notario_2004} a Vance Morrisona~\cite{Morrison_2008}. Konkrétně pro nás jsou důležité dva body.

\begin{enumerate}
    \item K \textit{inlinování funkce} nedojde, pokud se jedná o virtuální volání funkce.
    \item K \textit{inlinování funkce} nedojde, pokud je kód dané funkce příliš velký.
\end{enumerate}

Je nutné zmínit, že tyto články jsou již starší a nyní jsou pravidla pro inlinování méně přísná. Například JIT je schopný v některých případech provést \textit{inlinování funkce} i přes to, že se jedná o virtuálního volání funkce.

\subsection{Reprezentace systémů}
Jak již bylo v minulé sekci naznačeno, každý systém bude mít tedy funkci, kterou bude volat na každou entitu. Zároveň každý systém iteruje jednotlivé entity jiným způsobem. Systémy tedy budeme reprezentovat tím, že si pořídíme dva typy.

\begin{enumerate}
    \item \textbf{\texttt{EntityProcessor}:} Tento typ bude mít na sobě již zmiňovanou funkci pro zpracování jedné entity. Nazvěme ji \verb|Process|.
    \item \textbf{\texttt{IECSSystem}:} Jedná se o rozhraní. Pro každou ECS knihovnu, kterou budeme chtít měřit, implementujeme třídu, která hude dědit od tohoto rozhraní. Toto rozhraní bude nabízet metodu \verb|Update| ve které implementujeme iteraci přes jednotlivé entity. Každá instance takového systému bude mít také instanci \verb|EntityProcessor|, jejíž metodu \verb|Process| při iteraci zavolá na jednotlivé entity.
\end{enumerate}

Tím jsme vyřešili různé reprezentace systému, ale ještě zbývá problém s \textit{inlinováním funkce}. Vyřešíme to tím, že jednotlivé \verb|EntityProcessor| budou struktury, které budou dědit od rozhraní \verb|IEntityProcesor|. Instance systému poté přijme typ konkrétního \verb|EntityProcessor| jako generický argument.

Nyní si vysvětlíme proč je pro nás důležité, že jednotlivé \verb|EntityProcessor| jsou struktury.

Při první konstrukci generického typu, který má hodnotový datový typ jako parametr se vytvoří specializace dané třídy, ve které je daný parametr nahrazen daným typem. Je nutné poznamenat, že k tomuto dochází pouze v případě dosazení hodnotového datového typu (struktury jsou hodnotové datové typy). V případě referenčních datových typů se vytváří pouze jedna jediná specializace ve které je daný argument nahrazen typem \verb|object|. Pro více informací o tomto procesu je možné nahlédnout do článku \textit{Generics in the runtime (C\# programming guide)}~\cite{GenericsInTheRuntime} od firmy Microsoft.

Při použití konkrétního systému, kterému předáme jako generický argument jeden z \verb|EntityProcessor|, dojde tedy k výše zmíněnému procesu. Konkrétně dojde k vytvoření specializace pro již zmiňovanou metodu \verb|Update|. Tato specializace iteruje jednotlivé entity a na každé z nich volá metodu \verb|Process| na konkrétním \verb|EntityProcessor|. Na tomto volání může JIT provést \textit{inlinování}.

Může se stát, že metoda \verb|Process| na konkrétním \verb|EntityProcessor| bude příliš velká a nedojde k jejímu \textit{inlinování}. Tomu můžeme napomoci s použitím atributu \verb|MethodImplOptions.AggressiveInlining|~\cite{MethodImplOptions}, který JITu napoví, že bychom chtěli konkrétní metodu \textit{inlinovat} při jakémkoliv volání. Jedním z efektů při použití tohoto atributu je, že dojde k navýšení velikosti kódu, při které je kód dané funkce považován za příliš velký pro \textit{inlinování}. Může se stát, že velikost kódu dané funkce bude i po použití tohoto atributu příliš vysoká. V takovém případě ale bude cena vykonání dané funkce výrazně vyšší než její volání. 

\subsection{ECSFactory}
Každá implementace abstrakční vrstvy bude mít několik typů, které bude muset implementovat. Ovšem pro snazší práci s abstrakční vrstvou by bylo lepší kdyby bylo možné celou konkrétní implementaci reprezentovat jedním typem. Tento typ by byl zodpovědný za konstrukci jednotlivých typů z konkrétní implementace abstrakční vrstvy.

Zavedeme si tedy abstraktní třídu \verb|ECSFactory|. Každá konkrétní implementace abstrakční vrstvy poskytne třídu, které bude dědit od \verb|ECSFactory|.

Vzniká nám tu problém a to v tom, že pokud chceme vytvořit instanci systému (třídy, která dědí od \verb|IECSSystem|), je nutné ji předat kromě typu konkrétního \verb|EntityProcessor| také typy jednotlivých komponent se kterými bude systém pracovat (musí je předat \textit{query}). Ovšem tato informace je redundantní jelikož typy těchto komponent jsou zřejmé z konkrétního \verb|EntityProcessor|, který předáváme. Tímto nám vznikají méně přehledné a zdlouhavé řádky kódu pro vytváření jednotlivých systémů. Například pokud bychom měli \verb|ArchSystem| jako typ systému ECS knihovny \textit{Arch}~\cite{Arch} a \verb|PathFollowSystem| jako \verb|EntityProcessor|, který pracuje nad komponentami \verb|Location| (obsahuje informaci o pozici entity), \verb|Movement| (obsahuje informace o pohybu entity) a \verb|PathFollow| (obsahuje informace o pohybu entity po cestě), tak konstrukce instance tohoto systému by vypadala takto: \texttt{new ArchSystem<PathFollowSystem, Location, Movement, PathFollow>()}.

Kromě výše zmíněného problému je také další problém v tom, že třída zodpovědná za konstrukci těchto systémů musí vědět o tom jaký \verb|EntityProcessor| pracuje s jakými komponentami. Také pokud bychom chtěli přidat nebo odebrat komponentu pro nějaký \verb|EntityProcessor|, tak bychom museli tuto změnu provést na více místech (na místě definice daného \verb|EntityProcessor| a na místě jeho konstrukce).

Abychom tyto problémy vyřešili, předáme zodpovědnost za konstrukci těchto systémů třídě \verb|ECSFactory|. Ta skrze konstruktor dostane datový typ systému. Tato třída bude obsahovat metodu \verb|CreateSystem|, která přijme instanci konkrétního \verb|EntityProcessor| a pomocí reflexe zkonstruuje konkrétní finální systém. Výše zmíněný příklad by tedy s tímto řešením vypadal takto: \verb|factory.CreateSystem(new PathFollowSystem())|.

Je také nutné zmínit, že konkrétní implementace abstrakční vrstvy musejí poskytovat vícero tříd dědicích od \verb|IECSSystem|, jelikož je nutné definovat systém co pracuje s jednou komponentou, poté co pracuje s dvěma komponentami a tak dále. \verb|ECSFactory| tedy v konstruktoru přijímá více typů systému a při zavolání \verb|CreateSystem| vybere jeden z nich na základě počtu generických argumentů na typu \verb|EntityProcessor|.

\subsection{Komponenty jako třídy a komponenty jako struktury}
Jednotlivé ECS knihovny mají různé požadavky na typech jednotlivých komponent. Některé chtějí aby tyto typy byli třídy, jiné zase chtějí, aby tyto typy byly struktury a jiné povolují kombinaci obou variant.

Možné řešení by bylo mít dva typy pro každou komponentu, kde jeden by byl třída a jeden struktura. Ovšem toto řešení by vedlo k duplicitnímu kódu.

Řešení, které bylo použito na tento problém, je všechny typy komponent definovat jako struktury. V případě, že daná ECS knihovna vyžaduje aby jednotlivé typy komponent byly třídy, tak si definuje svou třídu \texttt{ComponentWrapper}, která jako generický argument přijme typ dané komponenty. Instance této třídy budou mít v sobě uloženou instanci konkrétní komponenty.

Některé ECS knihovny také požadují aby jednotlivé komponenty implementovali specifické rozhraní nebo atribut. V případě těchto knihoven nám postačí toto rozhraní nebo atribut implementovat na příslušný \texttt{ComponentWrapper} (ten nemusí být nutně třídou).

\section{Umělá inteligence}
Jak již víme ze sekce~\ref{subsec:villages}, naše hra bude obsahovat vesničany. Každý vesničan bude reprezentován entitou a bude vykonávat své akce na základě jednoduché umělé inteligence. Ovšem způsobů, kterými lze umělou inteligenci aneb chování dané entity definovat je více, proto si je nyní rozebereme. Mezi tyto způsoby patří:

\begin{enumerate}
    \item \textbf{Stavové automaty:} Nejpřímočařejší způsob je použít pro definici chování \textit{stavový automat}. Každý stav tohoto automatu reprezentuje činnost, kterou může entita provést. Jednotlivé stavy v sobě také mohou mít definované přechody, pomocí nichž mohou přejít do jiného stavu. \textit{Stavové automaty} bývají často reprezentovány pomocí návrhového vzoru \textit{State}, více o něm je možné se dočíst ve stejnojmenné kapitole v již zmiňované knize Game Programming Patterns~\cite{nystrom2014game}. Pro více informací o použití \textit{stavových automatů} pro umělou inteligenci je možné nahlédnout do materialu \textit{AI - Finite State Machines}~\cite{AIStateMachines} od \textit{Newcastle University}.
    
    \item \textbf{Stromy chování:} \textit{Stromy chování} jsou složeny z hierarchie vrcholů, která popisuje chování dané entity. Listy \textit{stromu chování} představují konkrétní příkazy, které řídí entitu. Ostatní vrcholy slouží k tomu aby řídili jak se strom bude procházet. Více o \textit{stromech chování} je možné se dočíst ve článku \textit{Behavior trees for AI: How they work}~\cite{BehaviorTrees} od Chrise Simpsona.
\end{enumerate}

Je nutné zmínit, že existuje mnoho dalších způsobů, kterými by se dalo chování entity definovat. Ovšem jak již bylo zmíněno chování našich vesničanů bude velmi jednoduché a pro jeho definování nám postačí tyto zmíněné způsoby.

Pro definici chování pro naše vesničany zvolíme \textit{stromy chování}. Hlavním důvodem pro tuto volbu je to, že se jedná o mnohem přehlednější způsob pro definování chování entit. Jelikož hierarchie \textit{stromu chování} popisuje chování dané entity, je možné přímo z definice této hierarchie vidět jak chování funguje. Jako příklad použijeme část ze stromu chování, který byl použit přímo pro naše vesničany:

\begin{verbatim}
FluentBuilder.Create<BehaviorContext>()
  .Selector("root")
    .Sequence("hunger")
      .Condition("is hunger bellow threshold?", IsHungry)
      .Condition("is there food?", IsThereFood(foodStockpile))
      .Do("go to get food", MoveTo(foodStockpile))
      // We need to check again because an another villager
      // could take the food in the meantime.
      .Condition("is there food?", IsThereFood(foodStockpile))
      .Do("get food", GetFood(foodStockpile))
      .Do("eat food", EatFood)
    .End()
  .End()
  .Build();
\end{verbatim}

% Umělou inteligenci je možné definovat také v systémech.

% Ne vždy je zřejmé co by mělo či nemělo být součástí umělé inteligence.

% Tezba surovin: připsání do inventáře vs damage and drop systém.

% Zpracování surovin: připsání do inventáře vs resource processing systém. 

% \subsection{ECS}
% Některé systémy proběhnou pro danou entitu pouze pokud je pro ní splněna určitá podmínka.

% Podmínka je drahá operace protože pipeline flush.

% Řešením mohou být eventy. EventQueue, EventReader/EventWriter, csharp events.

% Další možné řešení jsou reaction systémy.

% Proč eventy a reaction systémy nemůžeme použít. Proč nepoužíváme csharp eventy.

% Proč tyto systémy nezpůsobují výkonnostní potíže při správných podmínkách (+ ty podmínky).

\section{Procedurální terén}
\label{sec:terrain}
Naše hra bude obsahovat procedurálně generovaný terén. Ten je možné generovat pomocí CPU nebo pomocí GPU. Obě možnosti si nyní rozebereme.

% \subsubsection{Generování terénu na CPU}
Při generování terénu na CPU nám postačí terén vygenerovat pouze jednou při spuštění hry a výsledek si poté uložit do jedné nebo více textur. Tento přístup má ale dva problémy. Prvním problémem je, že pokud dojde k většímu přiblížení nebo oddálení kamery, dojde k deformaci těchto textur. To může způsobit různé vizuální artefakty. Druhým problémem je, že vytváření zmíněných textur může trvat delší dobu. To může vést k delšímu času spouštění hry.

% \subsubsection{Generování terénu na GPU}
Pro vygenerování terénu na GPU ja potřeba vytvořit shader. Tento shader vezme pozici a přiblížení kamery a vygeneruje pro tyto parametry terén. Díky vysoké paralelizace na GPU můžeme terén generovat za běhu. Při generování terénu za běhu nám nevzniknou výše zmíněné vizuální artefakty. Vyřeší se nám i druhý problém, jelikož nemusíme čekat na vygenerování výše zmíněné textury.

Je snadné nahlédnout, že generování terénu na GPU je pro nás výhodnější, proto tento přístup zvolíme.

\section{Hledání cesty ve vygenerovaném terénu}
\label{subsection:path-finding}
Vesničané musejí často vyhledávat cestu z bodu A do bodu B. Například potřebují dojít ke svému pracovnímu místu, nebo odnést předmět do skladiště.

V případě generování terénu na CPU by bylo hledání cesty velice jednoduché. Stačilo by nám vzít vygenerovanou texturu a najít danou cestu v ní. Ovšem problém spočívá v tom, že terén generujeme pomocí GPU a vyhledávání cesty musíme vykonávat na CPU.

Pro řešení problému s hledáním cesty si zavedeme \textit{compute shader}, který jako vstupní parametr přijme rozlišení. Výstupem tohoto \textit{compute shaderu} bude dvourozměrné pole sestavené z nevzorkovaného terénu při daném rozlišení. Tento \textit{compute shader} nám tedy při spuštění hry vytvoří dvourozměrné pole na GPU, to si poté stáhneme na CPU a v něm poté můžeme provádět vyhledávání cesty.

% Chceme pěkné cesty, takže výslednou cestu zkrášlujeme.

% Generování může trvat, protože přesun dat mezi GPU a CPU je drahý.

% Je nutné použít fork MonoGame, který podporuje compute shadery.

\section{Owner komponenta}
Při práci s ECS je někdy potřeba získat identifikátor příslušné entity. Ovšem v kontextu systému, který pouze iteruje přes komponenty na jednotlivých entitách to může být problematické.

Řešení, které některé ECS knihovny nabízejí je možnost iterovat identifikátory entit společně s jejich komponentami. Ovšem ne každá ECS knihovna to umí, proto je potřeba zvolit jiný přístup.

Přístup, který jsme zvolili je zavedení \texttt{Owner} komponenty. Jednotlivé entity jsou v naší hře reprezentovány třídou dědicí od \texttt{IEntity}. \texttt{Owner} komponenta má pouze jeden jediný člen a to referenci na \texttt{IEntity}, která představuje jejího majitele.









% \subsection{Reprezentace systémů}
% Jak již bylo zmíněno v úvodu, velký problém při návrhu systémů je rozdílné rozhrani, které pro systémy jednotlivé ecs knihovny nabízejí.

% Dále, jak již jsme si zmiňovali v podkapitole o herních požadavcích, chceme aby se výkon naší hry při použití této abstrakční vrstvy blížil výkonu s použitím pouze samotné ecs knihovny.

% Nahlédneme na oba tyto problémy blíže a prozkoumáme jak jsme je vyřešili.

% \subsubsection{Rozhraní ecs knihoven}
% ECS knihovny reprezentují systémy následujícími způsoby:

% \begon{ordering}
% \item \xxx{TODO}
% \end{ordering}











% V této sekci si představíme knihovny, které budeme chtít měřit. Nahlédneme na jejich rozhraní a na základě toho odůvodníme rozhodnutí provedené při implementaci abstrakční vrstvy.

% % - knihovny
% \subsection{Měřené knihovny}
% Než začneme provádět analýzu abstrakční vrstvy, musíme si nejprve představit jednotlivé knihovny, které budeme chtít měřit. Následně nahlédneme na jejich rozhraní, to nám pomůže při analýze abstrakční vrstvy.

% % --- predstaveni knihoven
% I přesto, že lze program použít pro měření velkého množství ECS knihoven pro C\#, omezíme se pouze na následující:

% \begin{enumerate}
%     \item Arch~\cite{Arch}
%     \item DefaultEcs~\cite{DefaultEcs}
%     \item Entitas~\cite{Entitas}
%     \item HypEcs~\cite{HypEcs}
%     \item LeoECS~\cite{LeoECS}
%     \item \xxx{Pridat dalsi knihovny kdyz bude cas.}
% \end{enumerate}

% \xxx{ --- rozebrat interface knihoven}\\\\

% \section{Inline entity processor}

% Inlinování funkcí je optimalizace, při které je volání funkce nahrazeno jejím obsahem.

% % priklad, kdy dochazi k inlinovani

% Kvůli měření je důležité, aby naše entity procesory neměli ideálně žádný overhead.

% Pro inlinování entity procesorů, využíváme následující techniky.

% % aggressive inlining, struktura + generika

% % \\
% % \xxx{Proc je to potreba}
% % \\
% % \xxx{Jak jsme toho dosahli}
% % \\

% \xxx{ - entity processor}\\\\
% \xxx{ --- -> entity processor}\\\\
% \xxx{ --- co je inline function call}\\\\
% \xxx{ --- jak jej zaridit}\\\\
% \xxx{ --- -> proto struktura a generika (IECSSystem)}\\\\

% \xxx{ - IEntity (chceme bohate rozhrani, jak jsme si popisovali)}\\\\
% \xxx{ - IWorld (nektere ECS knihovny potrebuji Update/Tick)}\\\\
% \xxx{ - Component attribute}\\\\
% \xxx{ - ECSFactory (proc types - zjednoduseni rozhrani)}\\\\

% \section{Analýza hry}

% text

% \subsection{Implementace AI}

% Nyní si rozebereme populární přístupy pro implementaci AI ve hrách.

% Nejpřímočařejší způsob je použít stavové automaty.

% Pokročilejší zpusob jsou behavior trees.

% Další pokročilejší způsob je Goal Oriented Action Planning (GOAP).

% Proč jsme zvolili behavior trees?

% Chování AI agentů může být popsané také přímo v systémech.

% Ne vždy je zřejmé co by mělo být součástí AI.

% \\
% \xxx{jaka jsou mozsnosti?}
% \\
% \xxx{- behavior trees}
% \\
% \xxx{- state machines}
% \\
% \xxx{- GOAP}
% \\
% \xxx{proc jsme zvolili behavior trees?}
% \\
% \xxx{behavior primo v systemech a proc jsme to nepouzili}
% \\
% \xxx{Co by melo byt rizeno AI a co ne?}
% \\
% \xxx{- tezba surovin: pripsani do inventare vs damage and drop system}
% \\
% \xxx{- zpracovani surovin: pripsani do inventare vs resource processing system}
% \\

% \subsection{ECS}

% Některé systémy v ECS reagují pouze pokud je pro danou entitu splněna určitá podmínka.

% Na první pohled by se mohlo zdát, že jsou tyto systémy špatné, jelikož podmínka je drahá operace.

% Alternativní a pravděpodobně lepší způsob je použít eventy.

% Rozeberme si jak by se s eventy v ECS pracovalo.

% Další způsob, jak by se tento problém dal řešit jsou Reaction systémy.

% \\
% \xxx{condition checking systems are ok}
% \\
% \xxx{eventy v ECS}
% \\

% \subsection{Generovani terenu}

% Rozeberme si dva přístupy pro generování terénu.

% První způsob je generovat terén přímo na CPU.

% Druhý způsob je pro generování terénu použít GPU.

% problem s path finding

% Abychom si rozebrali řešení, bude nejprve nutné nejprve říci něco více o shaderech.

% Compute shader můžeme použít k řešení výše zmíněného problému.

% \xxx{GPU vs CPU}