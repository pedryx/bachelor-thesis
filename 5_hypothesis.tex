\chapter{Hypotéza}
\label{chap:hypothesis}
V této kapitole si řekneme jak předpokládáme, že vyjdou naše experimenty.

\section{Cache}
Prvně si budeme muset přiblížit jak funguje cache protože to má velký vliv na naše experimenty.

Popsat paměťový model.

Přístup k datům.

Model je zjednodušen.

\section{Způsoby ukládání komponent}
Dále je nutné si rozebrat jak se běžné ukládají komponenty. Používají se běžně 3 způsoby.

\begin{enumerate}
    \item arch type

    popis

    jak ukládá komponenty
    \item sparse set

    popis

    jak ukládá komponenty

    \item komponenty jako třídy

    jak ukládají komponenty

    proč je to špatné
\end{enumerate}

\section{Hypotéza}
Nejrychleji vyjdou arch type protože nejlépe využívají cache.

Poté budou sparse set protože mají malí počet cache miss ale mají ho větší než arch type.

Nejhůře budou ty co chtějí aby komponenty byli třídy protože mají nejvíc cache miss.

% V této kapitole si rozebereme, jaké výsledky očekáváme od našeho měření.

% Relativní výkon bude stejný jako u Ecs.CSharp.Benchmark.

% Přiblížení cache?

% Lépe vyjdou knihovny, které jsou založeny na archtype nebo sparse set.

% Naopak knihovny, které dovolují pouze referenční datové typy pro komponenty vyjdou hůře.

% \\
% \xxx{Vysledek bude podobny jednoduchym testum z Ecs.CSharp.Benchmark.}
% \\
% \xxx{Vyjdou lepe ty knihovny co pouzivaji sparse set nebo archtype.}
% \\
% \xxx{Popsat co delaj spatne ty co vyjdou hure, napr pracuji s pointery na komponenty -> horsi vyuziti cache}
% \\