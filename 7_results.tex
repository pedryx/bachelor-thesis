\chapter{Experimenty a výsledky}
V této kapitole si nejprve představíme jak provádíme náš experiment. Následně budeme prezentovat jeho výsledky.

\section{Popis experimentů}
Shrneme si jak vlastně provádíme měření.

Zvládneme měřit velký počet knihoven ale nyní se zaměříme pouze na následující, kterými se zabýval původní test a jsou také populárními ecs knihovnami pro csharp.

\begin{enumerate}
    \item Arch~\cite{Arch}
    \item DefaultEcs~\cite{DefaultEcs}
    \item Entitas~\cite{Entitas}
    \item HypEcs~\cite{HypEcs}
    \item LeoECS~\cite{LeoECS}
    \item \xxx{Další knihovny???}
\end{enumerate}

\section{Výsledky}
Následující přehled zobrazuje jednotlivé ECS knihovny a jejich výsledky.

% proc je vic cache miss nez pocet arch type

\section{Analýza výsledků???}













% V této kapitole si prezentujeme experimenty a jejich výsledky.

% Shrneme si jak vlastně provádíme měření.

% Následující přehled zobrazuje jednotlivé ECS knihovny a jejich výsledky.
% % popsat prehled
% % co vysledky znamenaji

% \section{Analýza výsledků}
% Nyní si stručně rozebereme jednotlivé ECS knihovny a stručně si řekneme proč se umístili tak jak se umístili.

% \xxx{Proc vysli takoveto vysledky}
% \\
% \xxx{Strucne rozebrat jednotlive knihovny a strucne popsat proc se umistili lepe/hure}