\chapter{Experimenty a výsledky}
V této kapitole si nejprve shrneme jak provádíme náš experiment. Následně budeme prezentovat jeho výsledky.

\section{Popis experimentů}
V sekci~\ref{sec:performance-measurements} bylo řečeno, že měříme průměrný čas trvání jedné iterace \textit{game loop}. Ovšem ve skutečnosti měříme celkové trvání pevného počtu iterací. Výsledek je stejný, akorat je větší o násobek počtu iterací. To nám ale nevadí, jelikož tento počet iterací je stejný pro každou měřenou knihovnu.

Před samotným měřením provádíme přípravu. Během přípravy vytvoříme instanci hry, kterou následně inicializujeme a poté necháme běžet pevný počet iterací. Poté co doběhne příprava tak probíhá samotné měření, při kterém  necháváme hru běžet pevný počet iterací (ne nutně stejný jako při přípravě).

Zvládneme měřit velký počet knihoven, ale vytvářet porovnání budeme pouze z knihoven, které jsou něčím zajímavé. Mezi důvody zařazení patří například velký/malý výkon nebo popularita. Konkrétně budeme měřit výkon u následujících knihoven:

\begin{enumerate}
    \item Arch~\cite{Arch}
    \item DefaultEcs~\cite{DefaultEcs}
    \item Entitas~\cite{Entitas}
    \item FrifloEngineEcs~\cite{FrifloEngine}
    \item HypEcs~\cite{HypEcs}
    \item LeoECS~\cite{LeoECS}
    \item LeoEcsLite~\cite{LeoEcsLite}
    \item MonoGame.Extended.Entities~\cite{MonoGameExtended}
    \item RelEcs~\cite{RelEcs}
\end{enumerate}

V sekci~\ref{sec:abstract-layer} jsme si říkali, že konkrétní ECS knihovna je reprezentována třídou implementující rozhraní \texttt{IECSFactory}. Tato třída je při spuštění hry předána jako generický parametr třídě \texttt{Game}. Tím dojde k výběru konkrétní ECS knihovny.

Náš test přijímá jako parametr typ, který je právě výše zmíněná třída. Náš test tedy postupně spouštíme z třídami, které reprezentují jednotlivé ECS knihovny.

\section{Výsledky}
Následující přehled zobrazuje jednotlivé ECS knihovny a jejich výsledky.

% proc je vic cache miss nez pocet arch type

\section{Analýza výsledků???}













% V této kapitole si prezentujeme experimenty a jejich výsledky.

% Shrneme si jak vlastně provádíme měření.

% Následující přehled zobrazuje jednotlivé ECS knihovny a jejich výsledky.
% % popsat prehled
% % co vysledky znamenaji

% \section{Analýza výsledků}
% Nyní si stručně rozebereme jednotlivé ECS knihovny a stručně si řekneme proč se umístili tak jak se umístili.

% \xxx{Proc vysli takoveto vysledky}
% \\
% \xxx{Strucne rozebrat jednotlive knihovny a strucne popsat proc se umistili lepe/hure}