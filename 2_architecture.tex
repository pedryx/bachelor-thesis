\chapter{Architektury her}
V této kapitole si rozebereme architektury, které lze využít pro vývoj her. Pomůže nám to lépe porozumět problémům, které ECS řeší. Začneme objektovým návrhem a prozkoumáme problémy spojené s dědičností při vývoji her. Následně se zaměříme na návrhový vzor Component, který se snaží řešit tuto problematiku, a nahlédneme proč komunikace mezi komponentami může být problematická.

% Iterativni proces
Proto abychom lépe porozuměli herním architekturám, bude nutné si nejprve přiblížit jak se tvorba her liší od tvorby jiného typu softwaru. Vývoj her je iterativní proces, při kterém se nejprve navrhnou herní mechaniky. Následně probíhá testovaní, při kterém se zjistí, jestli jsou příslušné herní mechaniky zábavné. Pokud nejsou je nutné je upravit a znovu otestovat. Tyto dvě aktivity se opakují dokud vytvořené mechaniky nejsou dostatečně zábavné. V moment co jsou mechaniky vytvořené a víme o nich, že jsou zábavné, je možné vytvořit další mechaniky a celí proces opakovat. Takto se postupuje dokud nedojde k vytvoření celé hry.

% Co to implikuje?
Výše zmíněný iterativní proces, nám implikuje to, že zdrojový kód pro hry se mění velice často. Z toho důvodu je flexibilita velice důležitá vlastnost architektury při vývoji her.

\section{Objektový návrh}
% prirozenym zpusobem vyuzit bezneho objektoveho navrhu programovaciho jazyka, cili udelat nejakou sadu objektu provazanych dedicnosti
Přirozeným přístupem pro tvorbu software je použítí objektový návrh a s jeho pomocí si vytvořit sadu tříd provázaných dědičností. Ovšem jak již bylo naznačeno, v případě tvorby her může být tento přístup problematický.

% Jak se pouziva objektovy nabrh ve vyvoji her

Navrhneme si jednoduchou hru, na které si přiblížíme problémy s dědičností, které mohou při vývoji her s objektovým návrhem nastat. Budeme navrhovat jednoduchý tower defense \xxx{link}. Začneme třídou \verb|GameObject|, ta nám bude sloužit jako předek pro libovolný objekt ve scéně. Nyní bychom chtěli vytvořit jednotky, vytvoříme tedy třídu \verb|Unit|, která podědí od \verb|GameObject|. Chtěli bychom aby ve hře byli také speciální jednotky, které umí střílet, např. lučišníci. Vytvoříme tedy třídu \verb|Archer|, která dědí od třídy \verb|Unit|. Dále v naší hře budem potřebovat věže, uděláme tedy třídu \verb|Tower|, dědicí od třídy \verb|GameObject|. Určitě budem chtít od většiny našich věží aby uměli střílet, implementujeme tedy další třídu \verb|ShootingAbleTower|, dědicí od třídy \verb|Tower|. Do této třídy budem určitě chtít implementovat mechaniku střelby. Ovšem střelbu jsme již určitě implementovali do třídy \verb|Archer|, což nám nyní vytváří duplicitní kód.

Jak tento problém vyřešit? Prvním přístupem je využití multiple inheritance \xxx{link} (vícenásobné dědičnosti). Pokud programovací jazyk povoluje multiple inheritance, umožňuje to jedné třídě mít více než jednoho předka. Pokud bychom chtěli pomocí multiple inheritence vyřešit náš problém, mohli bychom implementovat třídu \verb|ShootingAble|, která dědí od \verb|GameObject|. Do této třídy bychom implementovali mechaniku střelby. Náš \verb|Archer| by poté dědil od \verb|ShootingAble| a od \verb|Unit|. Naše \verb|ShootingAbleTower| by dědila od \verb|ShootingAble| a od \verb|Tower|.

Tímto řešení by nám ale vznikl tzv. diamond problém \xxx{link}. Jedná se problém, který může vzniknout při používání multiple inheritance. Diamond problém nástává tehdy, pokud je potřeba přisoupit ke členům definovaných na společném předkovi. V našem případě se jedná o členy na třídě \verb|GameObject|.

Alternativním přístupem by mohlo být přesunutí společné logiky do společného předka. V našem případě bychom mechaniku střelby implementovali přímo ve třídě \verb|GameObject|.

Ovšem tímto porušíme jeden z principů objektového programování, konkrétně single responsibility \xxx{link}. Tento princip nám říká, že každá třída by měla být zodpovědná pouze za jednu věc. Porušení tohoto principu může mít za následek, že se nám ve společném předkovi začne hromadit velké množství logiky. To může vyústit v obrovskou třídu, která řeší všechny možné herné mechaniky. Jedná se o tzv. blob object \xxx{link}. 

Další problém je, že náš návrh je příliš závislý na hierarchii dědičnosti, což omezuje design naší hry. V praxi to znamená, že např. není možné za běhu přidat objektu schopnost střílet.

Řešením celé této situace je koncept composition over inheritance \xxx{link}. Tento princip nám říka, že namísto hluboké hierarchie dědičnosti bychom měli preferovat rozdělení objektů do komponent. Je to princip, na kterém je postavený návrhový vzor Component a také ECS.

\xxx{Ve skutečnosti byl princip composition over inheritance zmíněný již v Gang of Four a jedná se o dobrou praktiku v objektovém programování. - Byla by tohle dobrá poznámka pod čarou? }

\section{Návrhový vzor Component}
\label{sec:component}
Component je návrhový vzor... .

Tento návrhový vzor je velice populární.

Nyní si rozeberme problemy, které mohou vzniknout při používání návrhového vzoru Component.

Příklad?