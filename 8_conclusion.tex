\chapter{Závěr}
V této kapitole si připomeneme cíle, zhodnotíme co jsme mohli udělat lépe a řekneme si co jsou možná budoucí vylepšení.

\section{Cíle}
Připomeňme si znovu, jaké byli vlastně cíle naší práce.

Cíl h1 se nám povedlo splnit.

Pomocí této hry jsme porovnali výkon populárních ECS knihoven pro C\# a tím splnili cíl h2.

\section{Co jsme mohli udělat lépe?}
Nyní si rozebereme co bychom mohli udělat lépe.

Jedna z věcí co by mohla být lepší, je návrh kódu naší hry.

Hra také obsahuje několik bugů.

Další věc je, že některé části naší hry by mohli být jednodušší.

\section{Možná budoucí vylepšení}
Nyní si řekneme jaká jsou možná budoucí vylepšení.

Hra by mohla obsahovat interakce mezi vesnicemi (např souboje).

Mohli bychom přidat další měření co by měřilo čas jednotlivých systémů zvlášť, mohli bychom poté pozorovat jestli se pořadí knihoven bude lišit.

Dále bychom celé měření mohli provádět za běhu hry a dát možnost uživateli přepnout ecs knihovnu behem hraní.

% Připomeňme si znovu, jaké byli vlastně cíle naší práce.

% \section{Jak jsme splnili cíle práce?}
% Cíl h1 se nám povedlo splnit.

% Pomocí této hry jsme porovnali výkon populárních ECS knihoven pro C\# a tím splnili cíl h2.

% Díky návrhu naší hry, jsme splnili v1 a prozkoumali jaké výhody ma navrhovy vzor ECS.

% V kapitole 2, jsme si rozebrali architektury, které je možné použít pro vývoj her a ukázali jsme si, jaké výhody má návrhový vzor ECS s porovnáním s návrhovým vzorem Component. Tím jsme splnili cíl v2.

% \section{Co bychom mohli udělat lépe?}
% Jedna z věcí co by mohla být lepší, je návrh samotné hry.
% % kod + chyby

% Hra by mohla být jednodušší.

% \section{Možná budoucí vylepšení}
% Hra by mohla být komplexnější.

% Další možné vylepšení je podpora pro ECS knihovny závislé na herním enginu Unity.

% \xxx{Co jsme chteli a ceho jsme dosahli}
% \\
% \xxx{cp bychom mohli udelat lepe}
% \\
% \xxx{future work}