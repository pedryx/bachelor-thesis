\chapter{ECS}
\label{chap:ecs}
V této kapitole si více rozebereme návrhový vzor ECS. Řekneme si, jak se implementuje ukládání komponent, rozebereme si vlastnosti ECS a na závěr porovnáme ECS s návrhovým vzorem Component.

\section{Reprezentace ECS}
To jak ECS funguje jsme si již zmínili v úvodu. Nyní si rozebereme jak bývají jednotlivé prvky ECS reprezentovány při jejich implementaci. Krom základních členů, jako jsou entity, komponenty a systémy, si představíme ještě world a query.

Komponenty, jako takové, bývají často reprezentovány třídou nebo strukturou. Jak již bylo zmíněno v úvodu, komponenty obsahují pouze data, nikoliv žádnou logiku. Proto tyto třídy a struktury neobsahují žádné funkce.

Entita by, podobně jako v návrhovém vzoru Component, mohla být reprezentována jako kolekce svých komponent. Ovšem to se v praxi nedělá, namísto toho se používá řešení, které vede k lepšímu výkonu. Konkrétně se dělá to, že každá ECS implementace má správce všech entit a komponent, ve kterém jsou jednotlivé komponenty uloženy. Jednotlivé entity jsou potom reprezentovány pouze jako jednoduchý identifikátor a je úlohou tohoto správce aby mapoval jednotlivé entity k jejich komponentám.

Tomuto správci se v ECS implementacích běžně říká world. Jak již bylo zmíněno, slouží jako správce a kontejner pro všechny entity a komponenty v herním světě. Jeho rozhraní nabízí funkce, pomocí niž je možné vytvářet a mazat jednotlivé entity. Často obsahuje také funkce pro přidávání a odebírání komponent jednotlivým entitám.

Některé ECS implementace reprezentují entity pomocí malé struktury, obsahující identifikátor dané entity společně s referencí na world do kterého patří. To umožňuje mít na této struktuře bohaté rozhraní s funkcemi pro přidávání a odebírání komponent.

Reprezentace systémů se v jednotlivých ECS implementacích dost liší. Některé vyžadují aby systém byl třída dědicí od abstraktního předka, jiné zase volí opačný extrém a dovolují systém reprezentovat téměř jakkoliv. Navzdory velkým odlišnostem se ve velkém počtu ECS implementací často objevuje objekt, který mohou používat jednotlivé systémy pro iterování entit s určitou množinou komponent.

Tomuto objektu se v ECS implementacích běžně říká query. Pro jeho vytvoření je většinou nutné poskytnou informaci, nad jakými typy komponent bude query pracovat. Je například možné mít query, které pracuje nad komponentami Movement a Position. Takové query by pak bylo schopné iterovat nad všemi entitami co mají komponenty Movement a Position. Pomocí tohoto query by bylo možné implementovat MovementSystem, který by řídil logiku pohybu.

% Query se dělí do dvou typů, statické a dynamické. Statické query je možné zkonstruovat (znamenaje také specifikovat typy komponent nad kterými bude pracovat) pouze za kompilace. Naproti tomu dynamické query je možné zkonstruovat za běhu. Dynamické query je poměrně pokročilá věc a pouze malé množství ECS implementací jej podporuje.

% Je důležité si uvědomit, že správa entit a komponent je dost složitá úloha. Je nutné aby bylo možné namapovat jednotlivé komponenty na jejich entity. Zároveň je také zapotřebí aby bylo možné iterovat určité n-tice těchto komponent za pomocí query. Je také důležité brát v potaz to, že jednotlivé entity a komponenty se mohou přidávat a odebírat za běhu. Více si o těchto věcech řekneme v pozdější sekci, nyní se ale zaměřme na mazání entit. Jelikož smazání takové entity může být naročné na implementaci a nebo také na výkon, některé ECS implementace využívajicí koncept tombstone. Tombstone je oznaceni pro objekt, ktery po smazani stale existuje, pouze je někde poznamenáno, že je již smazaný. Smazané entity bývají tedy v některých ECS implementacích označeny jako tombstone a query je při iteraci ignoruje.

%\section{Správa paměti}
%Než si rozeberem jak je možné úkládat komponenty, bude nutné si přiblížit jak vlastně funguje cache v počítači a také jaký je rozdíl mezi hodnotovími a referenčními typy v C\#.

%Cache

%Referencni vs hodnotove typy

\section{Ukládaní komponent}
Nyní si rozebereme způsoby, kterými jednotlivé ECS implementace řeší ukládání komponent v paměti. Používají se převážně dva způsoby \textit{arch type} a \textit{sparse set}. Ovšem některé knihovny tyto způsoby nepoužívají a ukládají komponenty neefektivně. Přiblížíme si tedy také tento neefektivní způsob pro ukládání komponent.

\subsection{Arch type}
\xxx{Zkusit přeformulovat tuto subsekci.}

První způsob, který se používá pro ukládání komponent je tzv. \textit{arch type}. Funguje to tak, že se vezmou jednotlivé entity a rozdělí se do typů. Každému takovému typu se říká \textit{arch type}. \textit{Arch type} každé entity je určen množinou typů komponent ze kterých je složena. Každý \textit{arch type} je poté implementován jako kolekce, která má v sobě několik polí. Konkrétně pole pro každý typ komponenty. Každá entita v \textit{arch type} má poté přiřazen index. Komponenty na indexu i, náleží entitě s indexem i. Pokud poté vytvoříme query, které by mělo iterovat přes všechny entity s nějakou n-ticí komponent, tak toto query vezme vsechny \textit{arch type}, které obsahují danou n-tici a jednoduše proiteruje příslušné pole.

\xxx{obrazek}

Velkou nevýhodou \textit{arch types} je přidávání a odebírání komponent. Pokud je některé entitě přidána nebo odebrána komponenta, dojde ke změně jejího \textit{arch type}. V takovém případě je nutné překopírovat všechny komponenty příslušné entity z původního \textit{arch type} do nového. To může být výkonnostně náročná operace. Také je důležité si uvědomit, že odebrání prvku z pole je O(n) operace, ovšem tomuto se dá jednoduše předejít. Pokud nám nezáleží na pořadí prvků v poli, můžeme jednoduše vzít prvek, který chceme odebrat a prohodit ho s posledním prvkem v poli. Poté provedeme smazání posledního prvku. Tím dosáhneme smazání prvku z pole v O(1).

\subsection{Sparse set}

\xxx{velikost packed nám dává kapacitu, velikost dense max prvek - upravit popis aby toto odrážel}

\textit{Sparse set} je datová struktura, která reprezentuje množinu. Je možné do ní uložit čísla z množiny $0..u-1$. Na \textit{sparse set} je možné provádět operace vyhledání a vložení prvku v čase $O(1)$ a iteraci v čase $O(n)$. Rozbor této datové struktury je nad rámec této práce, proto pouze lehce nahlédneme na její implementaci.

\textit{Sparse set} se skládá ze dvou polí. Prvnímu se říká \textit{packed array} (někdy také \textit{internal array}, \textit{direct array}, nebo prostě \textit{values}). Druhé nese název \text{sparse array} (někdy také \textit{external array}, \textit{reverse array}, nebo prostě \textit{indices}). Obě tyto pole mají pevnou velikost. Pro \textit{packed array} je také stanovena kapacita, která odpovídá počtu prvků ve \textit{sparse set}. Příklad prázdného \textit{sparse set} pro $u=\text{\xxx{počet}}$ je možné vidět na obrázku \xxx{sparse_set_empty}. 

\xxx{obrázek - sparse_set_empty}

Při jakékoliv operace platí pro \textit{sparse set} následující invariant: $\forall v \in \text{sparse}: \text{packed}\left[\text{sparse}\left[v\right]\right] = v$. Jak je vidět na obrázku \xxx{sparse_set_empty}, tento invariant platí také pro prvky, které nejsou součástí \textit{sparse set}.

Nahlédněme prvně na operaci přidání prvku. Pokud bychom chtěli přidat prvek který se nachází hned za hranicí naší kapacity, konkrétně prvek $\text{packed}\left[\text{kapacita}\right]$ (v případě obrázku \xxx{sparse_set_empty} se jedná o prvek $0$), stačí nám zvýšit kapacitu o $1$. Tuto situaci můžeme vidět na obrázku \xxx{sparse_set_one}. V případě, že chceme přidat prvek, který se nachází na pozici $i$, kde $i > \text{kapacita}$, je nutné nejprve prohodit prvky $\text{packed}\left[\text{kapacita}\right]$ a $\text{packed}\left[i\right]$. Aby nedošlo k porušení invariantu, je také nutné prohodit odpovídající prvky v \textit{sparse array}. Po prohození prvků můžeme postupovat stejně jako v prvním případě. Tuto situaci můžeme vidět na obrázku \xxx{sparse_set_two}.

\xxx{obrázek - sparse_set_one}

\xxx{obrázek - sparse_set_two}

Přidejme si do našeho pole další dva prvky, konkrétně $2$ a $1$. Přidání provedeme v tomto pořadí. Současnou situaci je možné vidět na obrázku \xxx{sparse_set_four}. Nyní bychom chtěli nějaký prvek odebrat. V jednoduchém případě, kdy je náš prvek těsně před hranicí kapacity, konkrétně se jedná o prvek $\text{packed}\left[\text{kapacita - 1}\right]$ (v případě obrázku \xxx{sparse_set_four} prvek 1), stačí nám zmenšit kapacitu o $1$. Pokud se jedná o jiný prvek, tak podobně jako při odebírání, prohodíme tento prvek s prvkem $\text{packed}\left[\text{kapacita - 1}\right]$. Obdobně jako u odebíraní je nutné prohodit odpovídající prvky v \textit{sparse array} aby nedošlo k porušení invariantu. Poté postupujeme jako v prvním případě. Příklad druhé situace můžeme vidět na obrázku \xxx{sparse_set_three}.

\xxx{obrázek - sparse_set_three}

\xxx{Jak se používá sparse set v ECS.}

\subsection{Jiné způsoby ukládání komponent}
Komponenty jako pointery

\section{Vlastnosti ECS}
...

Vysoka flexibilita

Vysoky vykon

Flexibilita > vykon

\section{ECS vs Component}

\xxx{TODO}

\xxx{maybe topics: dynamicke vs staticke query, tombstones, relace, eventy, tag component, system types, reaction systems, paralelism, paralel queries, query filters, functional programming, relace, eventy, prukopnici (herni engine Bevy, ECS knihovny entitas a flecs)}