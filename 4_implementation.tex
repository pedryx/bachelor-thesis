\chapter{Popis Implementace}
V této kapitole si představíme klíčové a zajímavé části implementace naší hry. Pro více detailů o implementaci je možné nahlédnout do kódu, který obsahuje dokumentační komentáře.

Pro tvorbu hry byl použit MonoGame framework~\cite{MonoGame}, ovšem základní MonoGame nemá podporu pro compute shadery, které jsme využili pro řešení problému s vyhledávání cesty v podsekci~\ref{subsection:path-finding}. Z toho důvodu byl použit fork~\cite{MonoGameCptMax}, který tuto podporu nabízí.

\section{Abstrakční vrstva}
Některé typy z abstrakční vrstvy jsme si již představili v sekci \ref{section:abstract-layer-analysis}. Pro připomenutí si nyní tyto typy lehce představíme znovu a zároveň si popíšeme další důležité typy z abstrakční vrstvy:

\begin{enumerate}
    \item \textbf{\texttt{IECSSystem}:} Jedná se o rozhraní. Typy, které implementují toto rozhraní představují kompletní systém. Tyto typy obsahují logiku pro iteraci přes jednotlivé entity. O zpracování jednotlivých entit se stará \texttt{EntityProcessor}. Každá instance kompletního systému má na sobě právě jeden \texttt{EntityProcessor}. Například \texttt{ArchSystem<TEntityProcessor, TComponent>} je systém, který obsahuje logiku pro procházení entit z ECS knihovny \textit{Arch}~\cite{Arch}.

    \item \textbf{\texttt{IEntityProcesor}:} Jedná se o rozhraní, které ma na sobě klíčovou metodu \texttt{Process}. Typy, které implementují toto rozhraní obsahují herní logiku, která zpracovává jednotlivé entity. Například \verb|MovementSystem| ve své metodě \verb|Process| zpracuje pohyb pro danou entitu.

    Z pohledu hry jednotlivé implementace \textit{EntityProcessor} představují samotné systémy, jelikož hra nekouká na to jak jsou entity iterovány a zajímá je jenom samotná herní logika. Proto se kód, který pracuje s hrou odkazuje na \texttt{EntityProcessor} jako na systém. Ovšem z pohledu abstrakční vrstvy je nutné rozlišovat \texttt{System} jako typ, který obsahuje logiku pro iteraci entit a \texttt{EntityProcessor} jako typ, který obsahuje logiku pro zpracování entit. Autor si je vědom, že tyto pohledy mohou vést k matoucí terminologii, ovšem v kódu jsou již používány na příliš mnoho místech.

    \item \textbf{\texttt{IECSFactory}:} Jedná se o rozhraní. Typy, které implementují toto rozhraní definují logiku pro vytváření entity a \textit{world} dané implementace abstrakční vrstvy. Pomocí instancí těchto typů je také možné vytvářet instance kompletních systémů. Tyto typy také reprezentují celou jednu implementaci abstrakční vrstvy.

    \item \textbf{\texttt{IECSWorld}:} Jedná se rozhraní. Typy, které implementují toto rozhraní, představují \textit{world} pro danou ECS knihovnu. Jelikož pro velkou část ECS knihoven typy, které implementovali toto rozhraní, vypadali identicky, byl zaveden typ \texttt{BasicECSWorld}. Tento typ lze použít pro většinu ECS knihoven.

    \item \textbf{\texttt{IEntity}:} Jedná se o rozhraní. Typy, které implementují toto rozhraní představují entitu v konkrétní ECS knihovně.

    \item \textbf{\texttt{ComponentAttribute}:}. Některé ECS knihovny potřebují před spuštěním hry vědět o typech všech komponent, které budou ve hře používány. Proto typ jakékoliv komponenty, je označen tímto atributem a v případě, že některá ECS knihovna bude potřebovat znát typy jednotlivých komponent, může si skrze reflexi najít všechny typy s tímto atributem.
\end{enumerate}

\section{Páteřní kód}
Páteřní kód naší hry se skládá z následujících tříd:

\begin{enumerate}
    \item \textbf{\texttt{Game}:} Tato třída představuje samotnou hru. Dědí od třídy \texttt{Microsoft.Xna.Framework.Game } a přetěžuje od ní klíčové metody \verb|Update| a \verb|Draw|. Tato třída přijímá jako generický argument typ, který implementuje rozhraní \texttt{IECSFactory}. Pomocí tohoto typy je možné zvolit konkrétní implementaci abstrakční vrstvy neboli konkrétní ECS knihovnu se kterou ude hra pracovat.

    \item \textbf{\texttt{GameState}:} Abstraktní třída \verb|GameState| představuje herní obrazovku. Herní obrazovky můžou být například Menu, Nastavení, nebo Level. I přesto, že naše hra obsahuje pouze jednu herní obrazovku, díky této třídě by bylo jednoduché do hry výše zmíněné herní obrazovky přidat.

    \verb|GameState| si v sobě uchovává všechny systémy, které jsou rozděleny do dvou kolekcí. První je kolekce \verb|renderSystems|, která obsahuje všechny systémy, které řeší vykreslování. Druhá kolekce je \verb|systems|, která obsahuje všechny ostatní systémy.

    \verb|GameState| má na sobě definované dvě klíčové metody a to \verb|Update| a \verb|Draw|. V \verb|Update| se zpracují všechny systémy z kolekce \verb|systems| a v \verb|Draw| se zpracují všechny systémy z kolekce \verb|renderSystems|.

    \texttt{Game} má v jeden moment aktivní pouze jeden herní stav pro který ve své metodě \texttt{Update} volá \texttt{Update} na tomto herním stavu a ve své metodě \texttt{Draw} volá \texttt{Draw} na tomto herním stavu.

    Mezi Důležité členy, které \verb|GameState| obsahuje patří:

    \begin{enumerate}
        \item \textbf{\texttt{ECSWorld}:} Tento člen implementuje rozhraní \verb|IECSWorld| a jedná správce všech entit a komponent uvnitř jednoho \verb|GameState|.
        \item \textbf{\texttt{Camera}:} Představuje herní kameru.
        \item \textbf{\texttt{UILayer}:} Jedná se o správce prvků uživatelského rozhraní.
    \end{enumerate}

    \verb|GameState| obsahuje metodu \verb|Initialize| ve které jsou volány abstraktní metody \verb|CreateEntities|, \verb|CreateSystems|, \verb|CreateRenderSystems| a \verb|CreateUI|.

    \item \textbf{\texttt{LevelState}:} Třída \verb|LevelState| dědí od třídy \verb|GameState| a reprezentuje herní obrazovku, která obsahuje samotný level.

    \item \textbf{\texttt{LevelFactory}:} Třída \verb|LevelFactory| obsahuje metody pro vytváření jednotlivých entit, která jsou používané v \verb|LevelState|.
\end{enumerate}

\section{Herní svět}
Herní svět je reprezentován třídou \verb|GameWorld|. Tato třída obsahuje grid s informacemi o terénu, který je použit pro hledání cest. Do herního světa je možné přidávat suroviny skrze metodu \verb|AddResource|. Je také možné se dotazovat na informace ohledně terénu na konkrétní pozici v herním světě skrze metody \verb|GetTerrain|, \verb|IsWalkable|, \verb|IsBuildable|.

O generování herního světa se stará třída \verb|GameWorldGenerator|. Při generování světa se nejprve vygeneruje samotný terén, poté dojde ke generování surovin a následně ke generování vesnic.

Pro generování světa používáme dva shadery, konkrétně fragment shader pro vykreslený terénu definovaný v souboru \textit{terrainDraw.fx} a compute shader pro generování gridu definovaný v souboru \textit{terrainGen.fx}. Oba shadery využívají společnou část definovanou v souboru \textit{terrain.fx}.

Společná část definovaná v souboru \textit{terrain.fx} je zodpovědná za generování terénu. Pro generování terénu je použit simplexův šum na jehož generování je využita knihovna lygia~\cite{lygia}. Pomocí simplexova šumu je vygenerována výšková mapa, která je poté namapována v příslušné biomy. Ve společné části jsou definované dvě důležité funkce a to \verb|CalcNoise|, která spočítá výšku pro specifikovanou pozici a \verb|GetTerrain|, která přijme výšku a na základě ní navrátí informace o daném biomu, který je pro tuto výšku mapován.

Compute shader definovaný v souboru \textit{terrainGen.fx} přijímá vstupní parametry velikost světa vzdálenost mezi jednotlivými body v gridu. Mezi výstupy patří dva důležité buffery. Prvním je \verb|terrainBuffer|, který pro každý bod v gridu obsahuje identifikátor příslušného terénu. Druhým je \verb|resourceBuffer|, který obsahuje pozice na kterých bude vygenerována surovina.

Fragment shader definovaný v soubor \textit{terrainDraw.tx} je zodpovědný za vykreslení oblasti z herního světa. Mezi jeho vstupní parametry patří velikost světa, rozlišení, přiblížení kamery a pozice kamery. Tento shader je použit kromě vykreslování samotného herního světa také k vykreslení minimapy.

\section{Herní data a jejich reprezentace???}
% Popis co jsou herní data.

% Pro jejich správu používáme \verb|GlobalInstances|.

% Popis třídy.

% Herní data v naší hře.

% \begin{enumerate}
%     \item biomy
%     \item předměty
%     \item suroviny
% \end{enumerate}

\section{Vesnice a vesničané???}









% \chapter{Popis implementace}
% V této kapitole si blíže představíme implementaci naší hry.

% \section{Abstrakce}
% Prvně si rozebereme jak vypadá páteřní kód naší hry.

% % Game, GameState, LevelState, Factory, LevelFactory

% Nyní si přibližme naší abstrakční vrstvu Abstract.ECS.

% % \xxx{Code base}
% % \\
% % \xxx{Abstract.ECS layer}

% \section{Herní svět}
% Herní svět je reprezentován třídou GameWorld.

% O generování herního světa se stará třída GameWorldGenerator.

% Rozeberme si co jsou to vlastně shadery.

% Naše hra používá dva shadery.

% % \xxx{GameWorld}
% % \\
% % \xxx{GameWorldGenerator}
% % \\
% % \xxx{Shaders}

% \section{Herní data a jejich reprezentace}

% Pro správu herních dat, slouží třída GlobalInstances.

% % co jsou herni data, co trida dela

% % \xxx{Data - biomes, items, and resources}

% \section{Systémy a komponenty}

% Následující seznam obsahuje a popisuje všechny herní systémy a komponenty.