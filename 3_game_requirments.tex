\chapter{Co požadujeme od hry}
V této kapitole si rozebereme co požadujeme hry.

Připoměnme si prvně, že naší hru chceme použít pro měření ECS knihovny.

\section{Simulace}
Žánrem naší hry bude simulace, ale řekněmě si nejprve co to vlastně je.

% \\
% \xxx{Nestacilo by hodil citace na co je simulace a tenhle odstavec proste skiopnout?}
% \\

Proč jsme vlastně zvolili simulaci?

% \\
% \xxx{Proc jsme zvolili simulaci?}
% \\
% \xxx{- Velky pocet entit.}
% \\
% \xxx{- Mali vliv uzivatele na prubeh hry.}

\section{Charakteristiky hry}
Nyní si rozebereme jaké charakteristiky by naše hra měla splňovat.

Hra by měla mít velký počet entit.

Hra musí být dostatečně komplexní.

Hra nesmí být až příliš komplexní.

Hra by něměla využívat častého přidávání a odebírání komponent.

% \\
% \xxx{Jaka charakteristiky by nase hra mela mit?}
% \\
% \xxx{- Velky pocet entit.}
% \\
% \xxx{- Hra musi byt dostatecne slozita aby vysledek mel prakticky smysl.}
% \\
% \xxx{- Hra nesmi byt prislis slozita aby slozitost neodvadela od problemu.}
% \\
% \xxx{- Hra by nemela vyuzivat casteho pridavani/odebirani komponent.}
% \\
% \xxx{-- Kdyby ano, nektere knihovny by byli zvyhodnene.}