\chapter{Uvod}
Vytváření her je komplexní proces, který vyžaduje mnoho prototypování a iterací. Často se stává, že je potřeba změnit nemalou část již existujicího kódu. Z tohoto důvodu je klíčová vysoká flexibilita architektury herního kódu. Jedním ze způsobů, jak řešit tento problém, je použít návrhový vzor Entity-Component-System (ECS), který nejenže poskytuje vysokou flexibilitu, ale také vyniká výkonem. Jedním z cílů této práce bude zhodnotit výhody a nevýhody tohoto návrhového vzoru.

Implementace ECS je poměrně obtížná a časově náročná činnost, a proto mnoho vývojařů raději použije již existující knihovnu poskytující ECS. Existuje však velké množství těchto knihoven, ale naštěstí existují projekty, které nám mohou pomoci s výběrem takové knihovny. Velká část z těchto projektů provádí výkonnostní porovnání těchto knihoven na základě jednoduchých testů, avšak v případě rozsáhlejšího projektu, jako je hra, mohou být výsledky odlišné. Dalším cílem této práce proto bude vytvořit rozsáhlejší projekt (hru), který nám nejen pomůže při zhodnocení výhod a nevýhod ECS, ale také umožní srovnávat různé knihovny implementujicí tento návrhový vzor.

\section{Entity-Component-System}
Abychom mohli pokračovat, musíme si nejprve říct, co vlastně ECS je. Jedná se o návrhový vzor, který lze použít při návrhu her. Hlavním prvkem je entita, která reprezentuje objekt v herní scéně (např. hráče, strom, vesnici). Každá entita je složena z komponent, které obsahují data. Dále tu jsou systémy, které řídí logiku různých mechanik (např. pohyb, vykreslování, kolize). Při vhodném uspořádání komponent v paměti nám tento návrhový vzor může nabídnout velmi vysoký výkon.

\section{Existujicí ECS knihovny}


chceme udelat hru
existuje hodne ECS knihoven
implementace se hodne lisi
aby se dobre porovnavali bude potreba udelat abstract layer
bude to tezke protoze knihovny se hodne lisi


\iffalse

% proc bude navrh takoveto hry obtizny

\section{Existujicí ECS knihovny}
Na platformě .NET existuje nemalé množství knihoven kteří nabízejí ECS, bohužel nemálo z nich není nejlíp navrženo, proto se jimi budeme zajímat. Na internetu lze také nalézt několik projektů, které se tyto knihovny snaží porovnávat, ale žádná neporovnává tyho knihovny na konrkrétní hře.

\section{Cíle práce}
Cílem této práce je udělat hru pomocí ECS, která bude nezávislá na konkrétní ECS knihovně. Následně na této hře budeme porovnátat relativní výkon jednotlivých ECS knihoven. Dále chceme zhodnotit výhody a nevýhody ECS.

\xxx{Úvod je příliš krátký.}


\fi









\iffalse
\chapwithtoc{Introduction}

Introduction should answer the following questions, ideally in this order:
\begin{enumerate}
\item What is the nature of the problem the thesis is addressing?
\item What is the common approach for solving that problem now?
\item How this thesis approaches the problem?
\item What are the results? Did something improve?
\item What can the reader expect in the individual chapters of the thesis?
\end{enumerate}

Expected length of the introduction is between 1--4 pages. Longer introductions may require sub-sectioning with appropriate headings --- use \texttt{\textbackslash{}section*} to avoid numbering (with section names like `Motivation' and `Related work'), but try to avoid lengthy discussion of anything specific. Any ``real science'' (definitions, theorems, methods, data) should go into other chapters.
\todo{You may notice that this paragraph briefly shows different ``types'' of `quotes' in TeX, and the usage difference between a hyphen (-), en-dash (--) and em-dash (---).}

It is very advisable to skim through a book about scientific English writing before starting the thesis. I can recommend `\citetitle{glasman2010science}' by \citet{glasman2010science}.
\fi