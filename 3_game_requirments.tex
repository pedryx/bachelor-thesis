\chapter{Co požadujeme od hry}
V této kapitole si rozebereme požadavky, které požadujeme od naší hry. Prvně si rozeberem jednotlivé charakteristiky, které by naše hra měla splňovat. Dále na základě těchto charakteristik zvolíme žánr naší hry a odlvodníme proč jsme jej zvolili.

% Než budeme pokračovat, pripomeňmě si prvně, že naší hru chceme použít pro měření výkonu ECS knihoven.

\section{Charakteristiky hry}
Nyní si rozebereme jaké charakteristiky by naše hra měla splňovat. Zároveň si u každé odůvodníme proč je pro nás důležitá.

Hra by měla mít velký počet entit. Velkým počtem entit zajistíme, že jednotlivé systému budou vykonávat netriviální množství práce, celkové měření pro nás bude taky jednodušší.

Hra musí být dostatečně komplexní. Kdyby hra byla moc jednoduchá, tak jsme ničeho nedosáhli. Jedním z důvodů proč tuhle hru vůbec děláme je, že existujicí porovnání jsou příliš jednoduchá, proto je potrřeba aby naše hra byla dostatečna komplexní.

Hra nesmí být až příliš komplexní. Kdyby byla příliš komplexní, tak to odvádí od problému, který řešíme.

Hra by něměla využívat častého přidávání a odebírání komponent. Kdyby toho vuyžívala, tak by byli zvýhodněny sparse set based knihovny.

% \\
% \xxx{Jaka charakteristiky by nase hra mela mit?}
% \\
% \xxx{- Velky pocet entit.}
% \\
% \xxx{- Hra musi byt dostatecne slozita aby vysledek mel prakticky smysl.}
% \\
% \xxx{- Hra nesmi byt prislis slozita aby slozitost neodvadela od problemu.}
% \\
% \xxx{- Hra by nemela vyuzivat casteho pridavani/odebirani komponent.}
% \\
% \xxx{-- Kdyby ano, nektere knihovny by byli zvyhodnene.}

\section{Simulace}
Žánrem naší hry bude simulace, ale řekněmě si nejprve co to vlastně je. Simulace je žánr video her, kterž se snaší simulovat agenty v nějákem prostředí. U nás agenti budou vesničani a jejich prostředí bude příroda.

Proč jsme vlastně zvolili právě simulaci? Simulace nám umožní mít velký počet entit. Zároveň pro simulaci je snadné vymyslet herní design, který dobře škáluje s velikostí světa. To nám umožuje dělat jednoduše velké herní světy. Další velkou výhodou simulace je malí vliv od uživatele na průběh hry. Konkrétně v naší hře bude hráč spíše pozorovatel, který sleduje jak simulace probíhá.

% \\
% \xxx{Proc jsme zvolili simulaci?}
% \\
% \xxx{- Velky pocet entit.}
% \\
% \xxx{- Mali vliv uzivatele na prubeh hry.}