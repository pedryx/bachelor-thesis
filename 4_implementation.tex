\chapter{Popis Implementace}
V této kapitole si představíme klíčové a zajímavé části implementace naší hry. Pro více detailů o implementaci je možné nahlédnout do kódu, který obsahuje dokumentační komentáře.

Pro tvorbu hry byl použit MonoGame framework~\cite{MonoGame}, ovšem základní MonoGame nemá podporu pro compute shadery, které jsme využili pro řešení problému s \textit{path-finding} v podsekci~\ref{subsection:path-finding}. Z toho důvodu byl použit fork~\cite{MonoGameCptMax}, který tuto podporu nabízí.

\section{Páteřní kód}
Páteřní kód naší hry se skládá z následujících tříd:

\begin{enumerate}
    \item \textbf{Game:} Tato třída představuje samotnou hru. Dědí od třídy \verb|Microsoft.Xna.| \verb|Framework.Game| a přetěžuje od ní klíčové metody \verb|Update| a \verb|Draw|.

    \item \textbf{GameState:} Abstraktní třída \verb|GameState| představuje herní obrazovku. Herní obrazovky můžou být například Menu, Nastavení, nebo Level. I přesto, že naše hra obsahuje pouze jednu herní obrazovku, díky této třídě by bylo jednoduché do hry výše zmíněné herní obrazovky přidat.

    \verb|GameState| si v sobě uchovává všechny systémy, které jsou rozděleny do dvou kolekcí. První je kolekce \verb|renderSystems|, která obsahuje všechny systémy, které řeší vykreslování. Druhá kolekce je \verb|systems| a obsahuje všechny ostatní systémy.

    \verb|GameState| má na sobě definované dvě klíčové metody a to \verb|Update| a \verb|Draw|. V \verb|Update| se zpracují všechny systémy z kolekce \verb|systems| a v \verb|Draw| se zpracují všechny systémy z kolekce \verb|renderSystems|.

    Mezi Důležité členy, které \verb|GameState| obsahuje patří:

    \begin{enumerate}
        \item \textbf{ECSWorld:} Tento člen implementuje rozhraní \verb|IECSWorld| a jedná správce všech entit a komponent uvnitř jednoho \verb|GameState|.
        \item \textbf{Camera:} Představuje herní kameru.
        \item \textbf{UILayer:} Jedná se o správce prvků uživatelského rozhraní.
    \end{enumerate}

    \verb|GameState| obsahuje metodu \verb|Initialize| ve které jsou volány abstraktní metody \verb|CreateEntities|, \verb|CreateSystems|, \verb|CreateRenderSystems| a \verb|CreateUI|.

    \item \textbf{LevelState:} Třída \verb|LevelState| dědí od třídy \verb|GameState| a reprezentuje herní obrazovku, která obsahuje samotný level.

    \item \textbf{LevelFactory:} Třída \verb|LevelFactory| obsahuje metody pro vytváření jednotlivých entit, která jsou používané v \verb|LevelState|.
\end{enumerate}

%\section{Abstrakční vrstva???}

\section{Herní svět}
Herní svět je reprezentován třídou \verb|GameWorld|. Tato třída obsahuje grid s informacemi o terénu, který je použit pro \textit{path-finding}. Do herního světa je možné přidávat suroviny skrze metodu \verb|AddResource|. Je také možné se dotazovat na terén na konkrétní pozici v herním světě skrze metody \verb|GetTerrain|, \verb|IsWalkable|, \verb|IsBuildable|.

O generování herního světa se stará třída \verb|GameWorldGenerator|. Při generování světa se nejprve vygeneruje samotný terén, poté dojde ke generování surovin a následně ke generování vesnic.

Pro generování světa používáme dva shadery, konkrétně fragment shader pro vykreslený terénu definovaný v souboru \textit{terrainDraw.fx} a compute shader pro generování gridu definovaný v souboru \textit{terrainGen.fx}. Oba shadery využívají společnou část definovanou v souboru \textit{terrain.fx}.

Společná část definovaná v souboru \textit{terrain.fx} je zodpovědná za generování terénu. Pro generování terénu je použit simplexův šum na jehož generování je využita knihovna lygia~\cite{lygia}. Pomocí simplexova šumu je vygenerována výšková mapa, která je poté namapována v příslušné budovy. Ve společné části jsou definované dvě důležité funkce a to \verb|CalcNoise|, která spočítá výšku pro specifikovanou pozici a \verb|GetTerrain|, která přijme výšku a na základě ní navrátí informace o daném biomu, který je pro tuto výšku mapován.

Compute shader definovaný v souboru \textit{terrainGen.fx} přijímá parametry \verb|worldSize|, který představuje velikost světa a \verb|gridDistance|, který odpovídá vzdálenosti mezi jednotlivými body v gridu. Mezi výstup patří dva důležité buffery. Prvním je \verb|terrainBuffer|, který pro každý bod v gridu obsahuje identifikátor příslušného terénu. Druhým je \verb|resourceBuffer|, který obsahuje pozice na kterých bude vygenerována surovina.

Mezi vstupní parametry fragment shaderu patří velikost světa, rozlišení, scale a offset. Tento shader poté na základě těchto parametrů vygeneruje příslušný terén. Tento shader je krom vykreslování samotného terén použit také k vykreslení minimapy.

% \section{Herní data a jejich reprezentace}
% Popis co jsou herní data.

% Pro jejich správu používáme \verb|GlobalInstances|.

% Popis třídy.

% Herní data v naší hře.

% \begin{enumerate}
%     \item biomy
%     \item předměty
%     \item suroviny
% \end{enumerate}

% \section{Umělá inteligence???}

% \section{Systémy a komponenty???}









% \chapter{Popis implementace}
% V této kapitole si blíže představíme implementaci naší hry.

% \section{Abstrakce}
% Prvně si rozebereme jak vypadá páteřní kód naší hry.

% % Game, GameState, LevelState, Factory, LevelFactory

% Nyní si přibližme naší abstrakční vrstvu Abstract.ECS.

% % \xxx{Code base}
% % \\
% % \xxx{Abstract.ECS layer}

% \section{Herní svět}
% Herní svět je reprezentován třídou GameWorld.

% O generování herního světa se stará třída GameWorldGenerator.

% Rozeberme si co jsou to vlastně shadery.

% Naše hra používá dva shadery.

% % \xxx{GameWorld}
% % \\
% % \xxx{GameWorldGenerator}
% % \\
% % \xxx{Shaders}

% \section{Herní data a jejich reprezentace}

% Pro správu herních dat, slouží třída GlobalInstances.

% % co jsou herni data, co trida dela

% % \xxx{Data - biomes, items, and resources}

% \section{Systémy a komponenty}

% Následující seznam obsahuje a popisuje všechny herní systémy a komponenty.