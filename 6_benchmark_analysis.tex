\chapter{Analýza měření}
V této kapitole provedeme analýzu rozhodnutí, která byla vykonána během implementace měření. Prvně si rozebereme přístupy pro měření výkonu hry. Nakonec si řekneme, které věci při našem měření nebereme v potaz.

Pro měření výkonu jsme použili Benchmark.NET~\cite{BenchmarkDotNet}. Byl zvolen, jelikož se jedná o populární framework, který se běžně používá pro měření výkonu v aplikacích, které používají platformu .NET.

\section{Game loop}
Před rozborem jednotlivých přístupů si nejprve popíšeme co to je \textit{game loop} a představíme si pojmy, které s ním souvisí. Později se na ně budeme odkazovat při rozboru přístupů.

\textit{Game loop} si můžeme představit jako cyklus, který běží dokud nedojde k ukončení hry. Pseudokód pro \textit{game loop} může vypadat například takto:

\begin{verbatim}
  while (!shouldExit) 
  {
    Update();
    Draw();
  }
\end{verbatim}

Jak je možné nahlédnout, při každé iteraci \textit{game loop} dochází k volání funkcí \texttt{Update} a  \texttt{Draw}. \texttt{Update} se stará o zpracování herní logiky a \texttt{Draw} se stará o vykreslování.

V případě naší hry dojde při každém zavolání funkce \texttt{Draw} také k zavolání metody \texttt{Update} na všech systémech, které se starají o vykreslování. Podobně při každém zavolání funkce \texttt{Update} dojde také k zavolání metody \texttt{Update} na všech ostatních systémech.

\section{Přístupy k měření výkonu}
\label{sec:performance-measurements}
Nejpřímočařejší způsob pro měření výkonu hry je změřit jak dlouho trvá jedna iterace \textit{game loop}. Ovšem tento způsob je pro nás velice nevhodný, jelikož časy trvání každé iterace \textit{game loop} se mohou výrazně lišit.

Lepší způsob by bylo změřit čas trvání vícero iterací a následně spočítat jejich průměr. Při dostatečně velkém počtu iterací, tím získáme průměrný čas trvání jedné iterace \textit{game loop}. Toto je zároveň způsob, který jsme zvolili.

Dalším způsobem, kterým by lze měřit výkon, by bylo změřit průměrný čas trvání \texttt{Update} pro každý systém naší hry. Tento přístup by měl výhodu v tom, že by bylo možné nahlédnout na rychlost jednotlivých knihoven pro jednotlivé části hry. Ovšem náš zajímá celkový výkon, proto jsme tento přístup nezvolili.

Výkon hry by také bylo možné měřit během samotného hraní. To by nám pomohlo sledovat například to, jak se výkon hry mění v závislostí na akcích, která hráč provádí. Ovšem v naší hře je hráč pouze pozorovatelem a má velmi malí vliv na průběh hry, tím pádem pro nás tento přístup není moc užitečný.

\section{Věci, které při měření nebereme v potaz}
Nyní si řekneme jaké věci při měření nebereme v potaz a u každé si odůvodníme proč.

Mezi funkcionality, které některé ECS knihovny nabízejí patří takzvané \textit{reaction systémy}. Jedná se o speciální typ systému, který stejně jako jiné systémy pracují nad entitami, které obsahují žádoucí n-tici komponent. Ovšem tyto systémy v každé své iteraci zpracují pouze ty entity u kterých došlo ke změně na některé z žádoucí n-tice komponent.

V naší hře zmiňované \textit{reaction systémy} nepoužíváme. Problém je v tom, že jejich podporu nabízejí pouze některé ECS knihovny, tím pádem, pokud bychom je chtěli používat, nemohli bychom pak naši hru použít pro měření výkonu knihoven, které jejich podporu nenabízejí.

Kromě zmiňovaných \textit{reaction systémů} existuje i spoustu dalších funkcionalit, které některé ECS knihovny nabízejí pro zlepšení výkonu. My je ovšem v naší hře nepoužíváme a to ze stejného důvodu, kvůli kterému nepoužíváme \textit{reaction systémy}.
















% \chapter{Analýza měření}
% V této kapitole si rozebereme jaké existují přístupy k měření a na co při měření dbát ohled.

% \section{Problematika měření}
% Měření výkonu je komplexní proces, který by mohl být samostatnou prací sám o sobě.

% Benchmark.NET je populární framework pro platformu .NET, který lze použít k měření výkonu.

% \section{Přístupy k měření}
% Nejprve je nutné si říct co to je vlastně game loop.

% Nejpřímočařejší způsob je změřit jak dlouho trvá jedna iterace.

% Lepší způsob by tedy bylo změřit n iterací a spočítat jejich průměr.

% Rendering může být časově náročnější proces, rozeberme si jestli by měl či neměl být součástí měření.

% Dalším způsobem jak měřit výkon, je změřit iterace jednotlivých systémů.
% % proc to nedelame

% Výkon je také možné měřit během hraní samotné hry.
% % proc to nedelame

% % \xxx{Delta time jedne iterace}
% % \\
% % \xxx{Delta time n iteracich}
% % \\
% % \xxx{Mel by byt rendering soucasti mereni}
% % \\
% % \xxx{Mereni prumerneho delta time jednotlivych ssytemu}
% % \\
% % \xxx{Mereni behem hrani}

% \section{Na co při měření nebereme ohled?}
% Některé ECS knihovny pro získání lepšího výkonu používají paralelní queries.
% % co to je
% % nase hra je nepouziva protoze je z jednoduchosti navrzena single threaded

% Další častá optimalizace, co některé ECS knihovny nabízejí jsou reaction systémy.
% % co to je
% % nepouzivame to protoze to nemaji vsechny knihovny

% Jednotlivé ECS knihovny často používají další specifické výkonnostní funkcionality.
% % mi je nepouzivame protoze je nemaji vsechny knihovny

% Další věc na kterou při měření nebereme ohledy je přidávání a odebírání komponent.
% % sparse set muzou by zvyhodnene
% % nas herni design je prizpusoben tomu, ze to neovlivni mereni

% % \xxx{Paralelni queries}
% % \\
% % \xxx{Reaction systemy}
% % \\
% % \xxx{Pridavani/odebirani komponent za runtime - design prizpusoben}
% % \\
% % \xxx{jine vykonostni vylepseni specificke pro danou ECS knihovnu}